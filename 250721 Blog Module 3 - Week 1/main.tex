\documentclass[a4paper,12pt]{article}

% Encoding và font cho tiếng Việt
\usepackage{polyglossia}
\setmainlanguage{vietnamese}
\setotherlanguage{english}

\usepackage{fontspec}
\setmainfont{Times New Roman}
\setsansfont{Times New Roman}
\setmonofont{Times New Roman}[Scale=MatchLowercase]

% Toán học và ký hiệu
\usepackage{amsmath, amssymb, amsfonts}
\usepackage{pifont} % For \ding commands
\usepackage{bbding} % Alternative dingbat symbols

% Cấu hình trang
\usepackage{geometry}
\geometry{left=2.5cm, right=2.5cm, top=2.5cm, bottom=2.5cm}

% Header/Footer
\usepackage{fancyhdr}
\setlength{\headheight}{40 pt}
\pagestyle{fancy}
\fancyhead[LO,L]{GRID034 (AIO)}
\fancyhead[CO,C]{}
\fancyhead[RO,R]{\today}
\fancyfoot[CO,C]{\thepage}

% Hình ảnh
\usepackage{graphicx}
\usepackage{subcaption}

% Minted - mã lệnh có hỗ trợ tiếng Việt tốt hơn
\usepackage{minted}
\usepackage{float}

% Màu sắc cho minted
\usepackage{xcolor}
\definecolor{codebg}{rgb}{0.97,0.97,0.97}

% Cấu hình minted
\setminted{
    fontsize=\footnotesize,
    breaklines=true,
    breakanywhere=true,
    linenos=true,
    frame=lines,
    framesep=2mm,
    bgcolor=codebg
}

% Hyperref setup
\usepackage[hidelinks]{hyperref}
\hypersetup{
    colorlinks=true,
    linkcolor=blue,
    filecolor=magenta,
    urlcolor=cyan
}

% Các package bổ sung
\usepackage{parskip}
\usepackage{enumitem}
\usepackage{longtable}
\usepackage{booktabs}
\usepackage{makecell}
\usepackage{algorithm}
\usepackage{algpseudocode}
\usepackage{cancel}
\usepackage{titlesec}
\usepackage{blindtext}
\usepackage{datetime}
\usepackage{placeins}
\usepackage{array}
\usepackage{tabularx}
\usepackage{multirow}
\usepackage[table]{xcolor}
\usepackage[bottom]{footmisc}
\usepackage{tcolorbox}  % For the tcolorbox environment
\usepackage{tikz}
\usetikzlibrary{positioning, arrows.meta, shapes.geometric}

% Unicode special characters
\usepackage{newunicodechar}

% Custom command
\newcommand{\code}[1]{\texttt{#1}}

% Date format
\newdateformat{mydate}{\THEDAY/\THEMONTH/\THEYEAR}

% Section formatting
\titleformat{\section}{\large\bfseries}{\thesection.}{0.5em}{}

% Unicode characters
\newunicodechar{✔}{\checkmark}
\newunicodechar{❌}{\texttimes} % approximate
\newunicodechar{⇒}{\ensuremath{\Rightarrow}}

% Thông tin tài liệu
\author{}
\date{\mydate\today} % Gọi preamble từ file riêng

\begin{document}
\fancyhead[L]{%
    \includegraphics[height=1.2cm]{Logo/Logo.png}%
    }
\newpage

\begin{center}
    \vspace*{2em}
    {\LARGE\bfseries Blog Tuần 1 – Module 3 \par}
    \vspace{0.5em}
    {\Huge\bfseries Khám Phá Pandas, Trực Quan Hóa Dữ Liệu và ETL Pipeline \par}
    \vspace{1em}
    {\Large\itshape Từ nền tảng xử lý dữ liệu đến ứng dụng thực tiễn \par}
    \vspace{2em}
    {\large\textbf{Tác giả:} GRID034} \\
    \vspace{2em}
    \hrule
    \vspace{1.5em}
\end{center}

\noindent

\noindent
Tuần thứ nhất của Module 3 đưa ta vào một hành trình kiến thức đa dạng, kết hợp giữa lý thuyết nền tảng và ứng dụng thực tiễn. Nội dung Blog không chỉ giới thiệu các công cụ và kỹ thuật quan trọng trong phân tích dữ liệu, mà còn cung cấp các ví dụ minh họa trực quan và các tình huống thực tế giúp bạn áp dụng ngay vào công việc hoặc học tập. Từ việc làm chủ thư viện \textbf{Pandas}, lựa chọn biểu đồ trực quan phù hợp, xây dựng \textbf{ETL Pipeline} hiệu quả, cho tới kết hợp sức mạnh của \textbf{Python và Excel} – mọi phần đều được trình bày rõ ràng, dễ hiểu và bám sát thực tế.

\vspace{1em}

\noindent
Các chủ đề nổi bật bao gồm:

\vspace{1em}
\begin{enumerate}
    \item \textbf{Pandas – Từ lý thuyết đến ứng dụng thực tế}
    \begin{itemize}
        \item Giới thiệu \texttt{Series} và \texttt{DataFrame}, hai cấu trúc dữ liệu quan trọng.
        \item Thao tác cơ bản: đọc/ghi dữ liệu, xem thông tin, chọn lọc, thêm/xóa, sắp xếp, lọc, nhóm dữ liệu.
        \item Ứng dụng thực tế: phân tích dữ liệu Pokémon, xử lý dữ liệu thời gian, resampling và nhận diện seasonality.
    \end{itemize}

    \item \textbf{Trực quan hóa dữ liệu với Python}
    \begin{itemize}
        \item Giải thích khái niệm, vai trò, và nguyên tắc chọn biểu đồ phù hợp.
        \item Các thư viện phổ biến: \texttt{Matplotlib}, \texttt{Seaborn}, \texttt{Plotly}.
        \item Trường hợp nghiên cứu: Bộ dữ liệu \texttt{ETTh}, \texttt{Student Performance}, và \texttt{Iris} với nhiều loại biểu đồ (line chart, box plot, bar chart, donut chart, heatmap, scatter plot, bubble chart, pairplot, 3D visualization, word cloud).
    \end{itemize}

    \item \textbf{ETL Pipeline – Trái tim của dữ liệu thông minh}
    \begin{itemize}
        \item Quy trình \textbf{Extract – Transform – Load}.
        \item Kỹ thuật trích xuất dữ liệu (full, incremental, CDC), xử lý lỗi như composite key collision.
        \item Làm sạch và biến đổi dữ liệu, tạo cột mới, chuẩn hóa và tích hợp.
        \item Tải dữ liệu vào \texttt{Data Warehouse} hoặc \texttt{Data Mart}, xử lý thay đổi theo thời gian với \textbf{SCD Type 1 \& 2}.
    \end{itemize}

    \item \textbf{Kết hợp Python và Excel trong phân tích dữ liệu}
    \begin{itemize}
        \item So sánh ưu – nhược điểm giữa Excel và Python.
        \item Python trực tiếp trong Excel (hàm \texttt{=PY()}), áp dụng xử lý dữ liệu và vẽ biểu đồ ngay trong bảng tính.
        \item Lập trình Python bên ngoài để tự động hóa Excel: tạo file từ CSV, thêm định dạng, biểu đồ, và xử lý hàng loạt.
    \end{itemize}
\end{enumerate}

\vspace{1em}
\noindent\textbf{ Giá trị nhận được sau khi đọc Blog}
\begin{itemize}
    \item Hiểu và sử dụng thành thạo Pandas để xử lý dữ liệu.
    \item Lựa chọn biểu đồ trực quan phù hợp để truyền tải thông tin.
    \item Xây dựng và tối ưu hóa ETL Pipeline cho phân tích dữ liệu chuyên sâu.
    \item Kết hợp hiệu quả Python và Excel để tận dụng sức mạnh của cả hai công cụ.
\end{itemize}


\vspace{1em}
\newpage

\begin{center}
    \Large\textbf{Khám phá Gradient Boosting: Nghệ thuật điêu khắc mô hình từ sai số}
\end{center}

\begin{center}
    \Large\textit{Vũ Thái Sơn}
\end{center}

\begin{center}
    \large "Lấy sai số làm kim chỉ nam": Lý thuyết đến thực hành Gradient Boosting
\end{center}

\section{Giới thiệu: Ngoài việc học từ sai lầm, hãy học đúng hướng!}
\label{sec:gradient-boosting-advanced}

Hãy tưởng tượng bạn là một nhà điêu khắc đang tạc một bức tượng từ một khối đá cẩm thạch thô. Bạn không thể tạo ra kiệt tác ngay trong nhát đục đầu tiên. Thay vào đó, bạn bắt đầu với những đường nét cơ bản, sau đó lùi lại, quan sát những điểm chưa hoàn hảo -- "sai số" so với hình ảnh trong tâm trí -- rồi cẩn thận đục đẽo để sửa chữa những sai số đó. Mỗi lớp đá được loại bỏ là một bước đưa tác phẩm đến gần hơn với sự hoàn hảo.

Đây là triết lý cốt lõi của các thuật toán \textbf{Boosting} trong \textbf{Học tập Tập thể (Ensemble Learning)}. Thay vì xây dựng một mô hình phức tạp duy nhất, chúng ta kết hợp sức mạnh của nhiều mô hình đơn giản (weak learners), và mỗi mô hình sau sẽ học từ sai lầm của các mô hình trước. AdaBoost, một thuật toán tiên phong, thực hiện điều này bằng cách tăng trọng số cho các điểm bị dự đoán sai, buộc mô hình sau phải "chú ý" hơn đến chúng.

Nhưng Gradient Boosting còn tiến một bước xa hơn. Nó không chỉ hỏi "Tôi đã sai ở đâu?", mà còn hỏi \textbf{"Để sửa sai, tôi nên đi theo hướng nào là nhanh nhất?"}. Câu trả lời nằm ở hai chữ \textbf{"Gradient Descent"}. Gradient Boosting áp dụng một cách thiên tài ý tưởng tối ưu hóa này vào không gian của các mô hình, tạo ra một nghệ sĩ bậc thầy có khả năng điêu khắc nên những mô hình dự đoán với độ chính xác đáng kinh ngạc.

\begin{figure}[H]
    \centering
    \includegraphics[width=0.8\textwidth]{projects/GradientBoosting/images/boosting.png}
    \caption{Đi theo hướng Gradient để tìm loss bé nhất.}
    \label{fig:boosting_method}
\end{figure}

Trong bài viết này, chúng ta sẽ cùng nhau thực hiện một hành trình chi tiết, từ việc "giải phẫu" lý thuyết và công thức toán học, đến việc áp dụng nó vào hai ví dụ kinh điển: hồi quy và phân loại, với các bước tính toán cụ thể, và cuối cùng là hiện thực hóa bằng mã nguồn Python.

\section{Trái tim của thuật toán: "Gradient" đến từ đâu?}
\label{subsec:gradient-boosting-advanced-gradient}

Để thực sự hiểu Gradient Boosting, chúng ta cần nắm vững trực giác đằng sau chữ "Gradient". Hãy tưởng tượng bạn đang đứng trên một sườn núi trong sương mù, và mục tiêu là đi xuống thung lũng (nơi thấp nhất).

\begin{itemize}
    \item \textbf{Vị trí của bạn:} Tương đương với \textbf{dự đoán hiện tại} của mô hình.
    \item \textbf{Độ cao:} Tương đương với \textbf{sai số} (được đo bằng hàm mất mát - Loss Function).
    \item \textbf{Thung lũng:} Tương đương với \textbf{mô hình hoàn hảo} (sai số bằng 0).
\end{itemize}

Trong sương mù, bạn không thấy thung lũng. Cách duy nhất là cảm nhận \textbf{độ dốc (gradient)} ngay dưới chân mình. Gradient cho bạn biết hướng dốc lên nhất. Để đi xuống, bạn chỉ cần bước một bước nhỏ theo hướng \textbf{ngược lại với gradient (negative gradient)}. Lặp lại quá trình này, bạn sẽ dần dần đi tới đáy thung lũng.

Gradient Boosting đã "mượn" ý tưởng này. Nó xem việc tối ưu mô hình như một hành trình đi xuống "thung lũng sai số". \textbf{Phần dư (residual)} mà chúng ta sẽ tính toán ở các bước tiếp theo, về mặt toán học, chính là một xấp xỉ của \textbf{hướng negative gradient} đó. Mỗi cây quyết định mới được thêm vào chính là một "bước đi" theo hướng giảm sai số nhanh nhất.

\begin{figure}[H]
    \centering
    \includegraphics[width=0.8\textwidth]{projects/GradientBoosting/images/valley.png}
    \caption{Đi theo hướng Gradient để tìm loss bé nhất.}
    \label{fig:gradient_valley}
\end{figure}

\section{Gradient Boosting cho Bài toán Hồi quy}
\label{subsec:gradient-boosting-advanced-regression}
\subsection{Ý tưởng chính và Công thức}
Ý tưởng là xây dựng một chuỗi các cây quyết định, trong đó mỗi cây sau sẽ cố gắng dự đoán \textbf{phần dư (residual)} của tất cả các cây trước đó cộng lại.

\begin{enumerate}
    \item \textbf{Khởi tạo ($F_0$):} Bắt đầu với một dự đoán không đổi, thường là giá trị trung bình của biến mục tiêu $y$.
    \[ F_0(x) = \text{mean}(y) \]
    
    \item \textbf{Lặp (với $m$ từ 1 đến M):}
    \begin{enumerate}
        \item Tính phần dư (pseudo-residuals):
        \[ r_{im} = y_i - F_{m-1}(x_i) \]
        
        \item Huấn luyện một cây quyết định hồi quy $h_m(x)$ mới để dự đoán các phần dư $r_{im}$.
        
        \item Cập nhật mô hình tổng hợp:
        \[ F_m(x) = F_{m-1}(x) + \eta \cdot h_m(x) \]
    \end{enumerate}
\end{enumerate}

\subsection{Ví dụ tính toán chi tiết}
Hãy áp dụng quy trình trên vào bộ dữ liệu sau để dự đoán \textbf{Weight (kg)}.

\begin{figure}[H]
    \centering
    \includegraphics[width=0.8\textwidth]{projects/GradientBoosting/images/regression_dataset.png}
    \caption{Bộ dữ liệu ví dụ cho bài toán hồi quy.}
    \label{fig:regression_data}
\end{figure}

\subsubsection{Vòng lặp 1: Xây dựng cây đầu tiên}

\textbf{Bước 1: Khởi tạo ($F_0$)}
\[ F_0 = \frac{88+76+56+73+77+57}{6} = 71.17 \]

\textbf{Bước 2: Tính Phần dư ($r_1$)}
\begin{table}[H]
\centering
\begin{tabular}{rcccr}
\toprule
\textbf{ID} & \textbf{Height (m)} & \textbf{Weight (y)} & \textbf{$F_0$} & \textbf{Residual ($r_1$)} \\
\midrule
1 & 1.6 & 88 & 71.17 & 16.83 \\
2 & 1.6 & 76 & 71.17 & 4.83 \\
3 & 1.5 & 56 & 71.17 & -15.17 \\
4 & 1.8 & 73 & 71.17 & 1.83 \\
5 & 1.5 & 77 & 71.17 & 5.83 \\
6 & 1.4 & 57 & 71.17 & -14.17 \\
\bottomrule
\end{tabular}
\caption{Tính toán phần dư cho vòng lặp đầu tiên.}
\end{table}

\textbf{Bước 3: Xây dựng Cây $h_1(x)$}
Giả sử stump tốt nhất chia theo \textbf{Height $\leq$ 1.55}.
\begin{itemize}
    \item \textbf{Lá Trái (Height $\leq$ 1.55):} Gồm các điểm (ID 3, 5, 6).
    \[ \text{Leaf}_{\text{Left}} = \frac{-15.17 + 5.83 - 14.17}{3} \approx -7.84 \]
    \item \textbf{Lá Phải (Height > 1.55):} Gồm các điểm (ID 1, 2, 4).
    \[ \text{Leaf}_{\text{Right}} = \frac{16.83 + 4.83 + 1.83}{3} \approx 7.83 \]
\end{itemize}

\textbf{Bước 4: Cập nhật Mô hình ($F_1$)} \\
Giả sử tốc độ học $\eta = 0.1$.
\begin{table}[H]
\centering
\begin{tabular}{rccc}
\toprule
\textbf{ID} & \textbf{$F_0$} & \textbf{$h_1(x_i)$} & \textbf{$F_1(x_i) = F_0 + \eta \cdot h_1$} \\
\midrule
1 & 71.17 & 7.83 & $71.17 + 0.1 \times 7.83 = 71.95$ \\
2 & 71.17 & 7.83 & $71.17 + 0.1 \times 7.83 = 71.95$ \\
3 & 71.17 & -7.84 & $71.17 + 0.1 \times (-7.84) = 70.39$ \\
4 & 71.17 & 7.83 & $71.17 + 0.1 \times 7.83 = 71.95$ \\
5 & 71.17 & -7.84 & $71.17 + 0.1 \times (-7.84) = 70.39$ \\
6 & 71.17 & -7.84 & $71.17 + 0.1 \times (-7.84) = 70.39$ \\
\bottomrule
\end{tabular}
\caption{Cập nhật dự đoán sau vòng lặp đầu tiên.}
\end{table}

\subsubsection{Vòng lặp 2: Xây dựng cây thứ hai}
\textbf{Bước 5: Tính Phần dư mới ($r_2$)}
\begin{table}[H]
\centering
\begin{tabular}{rccc}
\toprule
\textbf{ID} & \textbf{Weight (y)} & \textbf{$F_1(x_i)$} & \textbf{Residual ($r_2 = y - F_1$)} \\
\midrule
1 & 88 & 71.95 & 16.05 \\
2 & 76 & 71.95 & 4.05 \\
3 & 56 & 70.39 & -14.39 \\
4 & 73 & 71.95 & 1.05 \\
5 & 77 & 70.39 & 6.61 \\
6 & 57 & 70.39 & -13.39 \\
\bottomrule
\end{tabular}
\caption{Tính toán phần dư cho vòng lặp thứ hai.}
\end{table}

\textbf{Bước 6: Xây dựng cây $h_2(x)$ và cập nhật $F_2(x)$} \\
Một cây quyết định mới, $h_2(x)$, sẽ được huấn luyện để dự đoán cột \textbf{Residual ($r_2$)}. Sau đó, mô hình sẽ được cập nhật một lần nữa: $F_2(x) = F_1(x) + \eta \cdot h_2(x)$.

\section{Gradient Boosting cho Bài toán Phân loại}
\label{subsec:gradient-boosting-advanced-classification}
\subsection{Ý tưởng chính và Công thức}
Thuật toán làm việc với xác suất và tối ưu trên không gian log-odds.

\begin{enumerate}
    \item \textbf{Khởi tạo ($F_0$):} Bắt đầu với một dự đoán log-odds không đổi.
    \[ F_0(x) = \log\left(\frac{p}{1-p}\right) \quad \text{với } p = \text{mean}(y) \]
    
    \item \textbf{Lặp (với $m$ từ 1 đến M):}
    \begin{enumerate}
        \item Chuyển log-odds $F_{m-1}(x)$ thành xác suất $p_{m-1}(x)$ bằng hàm Sigmoid.
        \item Tính phần dư giả (pseudo-residuals):
        \[ r_{im} = y_i - p_{m-1}(x_i) \]
        \item Huấn luyện cây $h_m(x)$ để dự đoán $r_{im}$.
        \item Cập nhật mô hình trong không gian log-odds:
        \[ F_m(x) = F_{m-1}(x) + \eta \cdot \gamma_m(x) \]
        (Với $\gamma_m(x)$ là giá trị đầu ra của các lá trong cây $h_m$).
    \end{enumerate}
\end{enumerate}

\subsection{Ví dụ tính toán chi tiết}
Áp dụng quy trình trên vào bộ dữ liệu sau để dự đoán \textbf{Loves Troll 2} (Yes=1, No=0).
\begin{figure}[H]
    \centering
    \includegraphics[width=0.8\textwidth]{projects/GradientBoosting/images/classification_dataset.png}
    \caption{Bộ dữ liệu ví dụ cho bài toán phân loại.}
    \label{fig:classification_data}
\end{figure}

\subsubsection{Vòng lặp 1: Xây dựng cây đầu tiên}

\textbf{Bước 1: Khởi tạo ($F_0, p_0$)}
\begin{itemize}
    \item Xác suất trung bình: $\bar{p} = 4/6 \approx 0.67$.
    \item Log-odds ban đầu: $F_0 = \log(\frac{0.67}{1-0.67}) \approx 0.71$.
    \item Xác suất dự đoán ban đầu: $p_0 = \text{sigmoid}(0.71) \approx 0.67$.
\end{itemize}

\textbf{Bước 2: Tính Phần dư giả ($r_1$)}
\begin{table}[H]
\centering
\begin{tabular}{rcccr}
\toprule
\textbf{ID} & \textbf{Age} & \textbf{Loves Troll 2 (y)} & \textbf{$p_0$} & \textbf{Residual ($r_1$)} \\
\midrule
1 & 12 & 1 & 0.67 & 0.33 \\
2 & 87 & 1 & 0.67 & 0.33 \\
3 & 44 & 0 & 0.67 & -0.67 \\
4 & 19 & 0 & 0.67 & -0.67 \\
5 & 32 & 1 & 0.67 & 0.33 \\
6 & 14 & 1 & 0.67 & 0.33 \\
\bottomrule
\end{tabular}
\caption{Tính toán phần dư giả cho vòng lặp đầu tiên.}
\end{table}

\textbf{Bước 3 \& 4: Xây dựng Cây $h_1(x)$ và Cập nhật $F_1(x)$} \\
Tương tự như bài toán hồi quy, một cây mới sẽ được tạo để học các phần dư này. Mô hình log-odds $F_0$ sẽ được cập nhật thành $F_1$, dẫn đến các xác suất dự đoán mới $p_1$ được cải thiện.

\section{Thực hành với Python}
\label{subsec:gradient-boosting-advanced-python}
\subsection{Mã nguồn cho Bài toán Hồi quy}
\begin{minted}{python}
import pandas as pd
import numpy as np
import matplotlib.pyplot as plt
from sklearn.ensemble import GradientBoostingRegressor
from sklearn.preprocessing import OneHotEncoder
from sklearn.compose import ColumnTransformer
from sklearn.pipeline import Pipeline

# 1. Create DataFrame from data
data_reg = {
    'Height': [1.6, 1.6, 1.5, 1.8, 1.5, 1.4],
    'Favorite Color': ['Blue', 'Green', 'Blue', 'Red', 'Green', 'Blue'],
    'Gender': ['Male', 'Female', 'Female', 'Male', 'Male', 'Female'],
    'Weight': [88, 76, 56, 73, 77, 57]
}
df_reg = pd.DataFrame(data_reg)

# 2. Prepare data for the model
X = df_reg.drop('Weight', axis=1)
y = df_reg['Weight']

# Define preprocessing steps
categorical_features = ['Favorite Color', 'Gender']
one_hot_encoder = OneHotEncoder(handle_unknown='ignore')
preprocessor = ColumnTransformer(
    transformers=[('cat', one_hot_encoder, categorical_features)],
    remainder='passthrough' # Keep other columns (Height)
)

# 3. Build and Train the Model
# Use GradientBoostingRegressor with basic parameters
# n_estimators: number of trees (boosting rounds)
# max_depth=1: each tree is a simple stump
gbr = GradientBoostingRegressor(n_estimators=100, learning_rate=0.1, max_depth=1, random_state=42)

# Create a pipeline to combine preprocessing and the model
model_pipeline = Pipeline(steps=[('preprocessor', preprocessor), ('regressor', gbr)])

# Train the pipeline on the entire dataset
model_pipeline.fit(X, y)

# 4. Visualize the results
plt.style.use('seaborn-v0_8-whitegrid')
plt.figure(figsize=(10, 6))
# Plot the actual data points
plt.scatter(df_reg['Height'], y, color='red', s=100, edgecolor='k', alpha=0.7, label='Actual Data')

# To plot the prediction line, create points and sort by Height
X_test_sorted = X.sort_values(by='Height')
y_pred_sorted = model_pipeline.predict(X_test_sorted)
plt.plot(X_test_sorted['Height'], y_pred_sorted, color='blue', linewidth=3, label='GBM Prediction')

# Customize the plot
plt.title('Gradient Boosting Regression: Height vs Weight', fontsize=16)
plt.xlabel('Height (m)', fontsize=12)
plt.ylabel('Weight (kg)', fontsize=12)
plt.legend()
plt.show()
\end{minted}

\begin{figure}[H]
    \centering
    \includegraphics[width=0.9\textwidth]{projects/GradientBoosting/images/regression_plot.png}
    \caption{Kết quả trực quan hóa của mô hình hồi quy GBM.}
    \label{fig:regression_plot}
\end{figure}

\subsection{Mã nguồn cho Bài toán Phân loại}
\begin{minted}{python}
import pandas as pd
import numpy as np
import matplotlib.pyplot as plt
from sklearn.ensemble import GradientBoostingClassifier
from sklearn.preprocessing import LabelEncoder
from mlxtend.plotting import plot_decision_regions

# 1. Create DataFrame
data_cls = {
    'Likes Popcorn': ['Yes', 'Yes', 'No', 'Yes', 'No', 'No'],
    'Age': [12, 87, 44, 19, 32, 14],
    'Favorite Color': ['Blue', 'Green', 'Blue', 'Red', 'Green', 'Blue'],
    'Loves Troll 2': ['Yes', 'Yes', 'No', 'No', 'Yes', 'Yes']
}
df_cls = pd.DataFrame(data_cls)

# 2. Preprocessing (simplified for 2D visualization)
le_popcorn = LabelEncoder()
df_cls['Likes Popcorn'] = le_popcorn.fit_transform(df_cls['Likes Popcorn'])
y_encoder = LabelEncoder()
df_cls['Loves Troll 2'] = y_encoder.fit_transform(df_cls['Loves Troll 2'])

# Select 2 features for visualization
X_vis = df_cls[['Age', 'Likes Popcorn']].values
y_vis = df_cls['Loves Troll 2'].values

# 3. Build and Train the Model
gbc = GradientBoostingClassifier(n_estimators=100, learning_rate=0.1, max_depth=1, random_state=42)
gbc.fit(X_vis, y_vis)

# 4. Visualize the decision boundary
plt.style.use('seaborn-v0_8-whitegrid')
plt.figure(figsize=(10, 6))
plot_decision_regions(X_vis, y_vis, clf=gbc, legend=2)
plt.title('Gradient Boosting Classification: Decision Boundary', fontsize=16)
plt.xlabel('Age', fontsize=12)
plt.ylabel('Likes Popcorn (0: No, 1: Yes)', fontsize=12)
plt.show()
\end{minted}

\begin{figure}[H]
    \centering
    \includegraphics[width=0.9\textwidth]{projects/GradientBoosting/images/classification_plot.png}
    \caption{Đường biên quyết định của mô hình phân loại GBM.}
    \label{fig:classification_plot}
\end{figure}

\section{Tinh chỉnh siêu tham số để đạt hiệu suất tối ưu}
\begin{itemize}
    \item \textbf{\texttt{n\_estimators}}: Số lượng cây trong chuỗi.
    \item \textbf{\texttt{learning\_rate} ($\eta$)}: Tốc độ học. Có một sự đánh đổi quan trọng giữa \texttt{learning\_rate} và \texttt{n\_estimators}.
    \item \textbf{\texttt{max\_depth}}: Độ sâu tối đa của mỗi cây. Với GBM, chúng ta thường ưu tiên các cây nông (1 đến 5).
    \item \textbf{\texttt{subsample}}: Tỷ lệ mẫu dữ liệu được sử dụng để huấn luyện mỗi cây (Stochastic Gradient Boosting).
\end{itemize}

\section{So sánh với các thuật toán Ensemble khác}
\begin{table}[H]
\centering
\caption{So sánh Gradient Boosting, Random Forest, và AdaBoost.}
\begin{tabular}{l|p{3.5cm}|p{3.5cm}|p{3.5cm}}
\toprule
\textbf{Tiêu chí} & \textbf{Gradient Boosting} & \textbf{Random Forest}~\cite{breiman2001random} & \textbf{AdaBoost} \\
\midrule
\textbf{Thứ tự} & Tuần tự & Song song & Tuần tự \\
\textbf{Cơ chế học} & Học trên phần dư (gradient) & Học trên mẫu bootstrap độc lập & Học trên trọng số của các điểm bị sai \\
\textbf{Mục tiêu} & Giảm bias trước, sau đó giảm variance & Chủ yếu giảm variance & Chủ yếu giảm bias \\
\textbf{Độ sâu cây} & Cây nông & Cây sâu & Cây rất nông (stumps) \\
\bottomrule
\end{tabular}
\label{tab:comparison}
\end{table}

\section{Tổng kết}
Gradient Boosting không chỉ là một thuật toán; nó là một framework mạnh mẽ và có tính tổng quát cao~\cite{friedman2001greedy}. Bằng cách hiểu rõ cơ chế học tập theo gradient, chúng ta không chỉ nắm vững một công cụ mạnh mẽ mà còn có một nền tảng vững chắc để tiếp cận các thuật toán "hậu duệ" của nó như XGBoost~\cite{chen2016xgboost}, LightGBM và CatBoost, những "gã khổng lồ" đang thống trị nhiều cuộc thi về khoa học dữ liệu hiện nay.

% Tạo danh mục tài liệu tham khảo từ file references.bib
\bibliographystyle{plain}
\bibliography{projects/GradientBoosting/references}
\clearpage
\setcounter{section}{0} % Reset section numbering

\begin{center}
    \Large\textbf{Khám phá Gradient Boosting: Nghệ thuật điêu khắc mô hình từ sai số}
\end{center}

\begin{center}
    \Large\textit{Vũ Thái Sơn}
\end{center}

\begin{center}
    \large "Lấy sai số làm kim chỉ nam": Lý thuyết đến thực hành Gradient Boosting
\end{center}

\section{Giới thiệu: Ngoài việc học từ sai lầm, hãy học đúng hướng!}
\label{sec:gradient-boosting-advanced}

Hãy tưởng tượng bạn là một nhà điêu khắc đang tạc một bức tượng từ một khối đá cẩm thạch thô. Bạn không thể tạo ra kiệt tác ngay trong nhát đục đầu tiên. Thay vào đó, bạn bắt đầu với những đường nét cơ bản, sau đó lùi lại, quan sát những điểm chưa hoàn hảo -- "sai số" so với hình ảnh trong tâm trí -- rồi cẩn thận đục đẽo để sửa chữa những sai số đó. Mỗi lớp đá được loại bỏ là một bước đưa tác phẩm đến gần hơn với sự hoàn hảo.

Đây là triết lý cốt lõi của các thuật toán \textbf{Boosting} trong \textbf{Học tập Tập thể (Ensemble Learning)}. Thay vì xây dựng một mô hình phức tạp duy nhất, chúng ta kết hợp sức mạnh của nhiều mô hình đơn giản (weak learners), và mỗi mô hình sau sẽ học từ sai lầm của các mô hình trước. AdaBoost, một thuật toán tiên phong, thực hiện điều này bằng cách tăng trọng số cho các điểm bị dự đoán sai, buộc mô hình sau phải "chú ý" hơn đến chúng.

Nhưng Gradient Boosting còn tiến một bước xa hơn. Nó không chỉ hỏi "Tôi đã sai ở đâu?", mà còn hỏi \textbf{"Để sửa sai, tôi nên đi theo hướng nào là nhanh nhất?"}. Câu trả lời nằm ở hai chữ \textbf{"Gradient Descent"}. Gradient Boosting áp dụng một cách thiên tài ý tưởng tối ưu hóa này vào không gian của các mô hình, tạo ra một nghệ sĩ bậc thầy có khả năng điêu khắc nên những mô hình dự đoán với độ chính xác đáng kinh ngạc.

\begin{figure}[H]
    \centering
    \includegraphics[width=0.8\textwidth]{projects/GradientBoosting/images/boosting.png}
    \caption{Đi theo hướng Gradient để tìm loss bé nhất.}
    \label{fig:boosting_method}
\end{figure}

Trong bài viết này, chúng ta sẽ cùng nhau thực hiện một hành trình chi tiết, từ việc "giải phẫu" lý thuyết và công thức toán học, đến việc áp dụng nó vào hai ví dụ kinh điển: hồi quy và phân loại, với các bước tính toán cụ thể, và cuối cùng là hiện thực hóa bằng mã nguồn Python.

\section{Trái tim của thuật toán: "Gradient" đến từ đâu?}
\label{subsec:gradient-boosting-advanced-gradient}

Để thực sự hiểu Gradient Boosting, chúng ta cần nắm vững trực giác đằng sau chữ "Gradient". Hãy tưởng tượng bạn đang đứng trên một sườn núi trong sương mù, và mục tiêu là đi xuống thung lũng (nơi thấp nhất).

\begin{itemize}
    \item \textbf{Vị trí của bạn:} Tương đương với \textbf{dự đoán hiện tại} của mô hình.
    \item \textbf{Độ cao:} Tương đương với \textbf{sai số} (được đo bằng hàm mất mát - Loss Function).
    \item \textbf{Thung lũng:} Tương đương với \textbf{mô hình hoàn hảo} (sai số bằng 0).
\end{itemize}

Trong sương mù, bạn không thấy thung lũng. Cách duy nhất là cảm nhận \textbf{độ dốc (gradient)} ngay dưới chân mình. Gradient cho bạn biết hướng dốc lên nhất. Để đi xuống, bạn chỉ cần bước một bước nhỏ theo hướng \textbf{ngược lại với gradient (negative gradient)}. Lặp lại quá trình này, bạn sẽ dần dần đi tới đáy thung lũng.

Gradient Boosting đã "mượn" ý tưởng này. Nó xem việc tối ưu mô hình như một hành trình đi xuống "thung lũng sai số". \textbf{Phần dư (residual)} mà chúng ta sẽ tính toán ở các bước tiếp theo, về mặt toán học, chính là một xấp xỉ của \textbf{hướng negative gradient} đó. Mỗi cây quyết định mới được thêm vào chính là một "bước đi" theo hướng giảm sai số nhanh nhất.

\begin{figure}[H]
    \centering
    \includegraphics[width=0.8\textwidth]{projects/GradientBoosting/images/valley.png}
    \caption{Đi theo hướng Gradient để tìm loss bé nhất.}
    \label{fig:gradient_valley}
\end{figure}

\section{Gradient Boosting cho Bài toán Hồi quy}
\label{subsec:gradient-boosting-advanced-regression}
\subsection{Ý tưởng chính và Công thức}
Ý tưởng là xây dựng một chuỗi các cây quyết định, trong đó mỗi cây sau sẽ cố gắng dự đoán \textbf{phần dư (residual)} của tất cả các cây trước đó cộng lại.

\begin{enumerate}
    \item \textbf{Khởi tạo ($F_0$):} Bắt đầu với một dự đoán không đổi, thường là giá trị trung bình của biến mục tiêu $y$.
    \[ F_0(x) = \text{mean}(y) \]
    
    \item \textbf{Lặp (với $m$ từ 1 đến M):}
    \begin{enumerate}
        \item Tính phần dư (pseudo-residuals):
        \[ r_{im} = y_i - F_{m-1}(x_i) \]
        
        \item Huấn luyện một cây quyết định hồi quy $h_m(x)$ mới để dự đoán các phần dư $r_{im}$.
        
        \item Cập nhật mô hình tổng hợp:
        \[ F_m(x) = F_{m-1}(x) + \eta \cdot h_m(x) \]
    \end{enumerate}
\end{enumerate}

\subsection{Ví dụ tính toán chi tiết}
Hãy áp dụng quy trình trên vào bộ dữ liệu sau để dự đoán \textbf{Weight (kg)}.

\begin{figure}[H]
    \centering
    \includegraphics[width=0.8\textwidth]{projects/GradientBoosting/images/regression_dataset.png}
    \caption{Bộ dữ liệu ví dụ cho bài toán hồi quy.}
    \label{fig:regression_data}
\end{figure}

\subsubsection{Vòng lặp 1: Xây dựng cây đầu tiên}

\textbf{Bước 1: Khởi tạo ($F_0$)}
\[ F_0 = \frac{88+76+56+73+77+57}{6} = 71.17 \]

\textbf{Bước 2: Tính Phần dư ($r_1$)}
\begin{table}[H]
\centering
\begin{tabular}{rcccr}
\toprule
\textbf{ID} & \textbf{Height (m)} & \textbf{Weight (y)} & \textbf{$F_0$} & \textbf{Residual ($r_1$)} \\
\midrule
1 & 1.6 & 88 & 71.17 & 16.83 \\
2 & 1.6 & 76 & 71.17 & 4.83 \\
3 & 1.5 & 56 & 71.17 & -15.17 \\
4 & 1.8 & 73 & 71.17 & 1.83 \\
5 & 1.5 & 77 & 71.17 & 5.83 \\
6 & 1.4 & 57 & 71.17 & -14.17 \\
\bottomrule
\end{tabular}
\caption{Tính toán phần dư cho vòng lặp đầu tiên.}
\end{table}

\textbf{Bước 3: Xây dựng Cây $h_1(x)$}
Giả sử stump tốt nhất chia theo \textbf{Height $\leq$ 1.55}.
\begin{itemize}
    \item \textbf{Lá Trái (Height $\leq$ 1.55):} Gồm các điểm (ID 3, 5, 6).
    \[ \text{Leaf}_{\text{Left}} = \frac{-15.17 + 5.83 - 14.17}{3} \approx -7.84 \]
    \item \textbf{Lá Phải (Height > 1.55):} Gồm các điểm (ID 1, 2, 4).
    \[ \text{Leaf}_{\text{Right}} = \frac{16.83 + 4.83 + 1.83}{3} \approx 7.83 \]
\end{itemize}

\textbf{Bước 4: Cập nhật Mô hình ($F_1$)} \\
Giả sử tốc độ học $\eta = 0.1$.
\begin{table}[H]
\centering
\begin{tabular}{rccc}
\toprule
\textbf{ID} & \textbf{$F_0$} & \textbf{$h_1(x_i)$} & \textbf{$F_1(x_i) = F_0 + \eta \cdot h_1$} \\
\midrule
1 & 71.17 & 7.83 & $71.17 + 0.1 \times 7.83 = 71.95$ \\
2 & 71.17 & 7.83 & $71.17 + 0.1 \times 7.83 = 71.95$ \\
3 & 71.17 & -7.84 & $71.17 + 0.1 \times (-7.84) = 70.39$ \\
4 & 71.17 & 7.83 & $71.17 + 0.1 \times 7.83 = 71.95$ \\
5 & 71.17 & -7.84 & $71.17 + 0.1 \times (-7.84) = 70.39$ \\
6 & 71.17 & -7.84 & $71.17 + 0.1 \times (-7.84) = 70.39$ \\
\bottomrule
\end{tabular}
\caption{Cập nhật dự đoán sau vòng lặp đầu tiên.}
\end{table}

\subsubsection{Vòng lặp 2: Xây dựng cây thứ hai}
\textbf{Bước 5: Tính Phần dư mới ($r_2$)}
\begin{table}[H]
\centering
\begin{tabular}{rccc}
\toprule
\textbf{ID} & \textbf{Weight (y)} & \textbf{$F_1(x_i)$} & \textbf{Residual ($r_2 = y - F_1$)} \\
\midrule
1 & 88 & 71.95 & 16.05 \\
2 & 76 & 71.95 & 4.05 \\
3 & 56 & 70.39 & -14.39 \\
4 & 73 & 71.95 & 1.05 \\
5 & 77 & 70.39 & 6.61 \\
6 & 57 & 70.39 & -13.39 \\
\bottomrule
\end{tabular}
\caption{Tính toán phần dư cho vòng lặp thứ hai.}
\end{table}

\textbf{Bước 6: Xây dựng cây $h_2(x)$ và cập nhật $F_2(x)$} \\
Một cây quyết định mới, $h_2(x)$, sẽ được huấn luyện để dự đoán cột \textbf{Residual ($r_2$)}. Sau đó, mô hình sẽ được cập nhật một lần nữa: $F_2(x) = F_1(x) + \eta \cdot h_2(x)$.

\section{Gradient Boosting cho Bài toán Phân loại}
\label{subsec:gradient-boosting-advanced-classification}
\subsection{Ý tưởng chính và Công thức}
Thuật toán làm việc với xác suất và tối ưu trên không gian log-odds.

\begin{enumerate}
    \item \textbf{Khởi tạo ($F_0$):} Bắt đầu với một dự đoán log-odds không đổi.
    \[ F_0(x) = \log\left(\frac{p}{1-p}\right) \quad \text{với } p = \text{mean}(y) \]
    
    \item \textbf{Lặp (với $m$ từ 1 đến M):}
    \begin{enumerate}
        \item Chuyển log-odds $F_{m-1}(x)$ thành xác suất $p_{m-1}(x)$ bằng hàm Sigmoid.
        \item Tính phần dư giả (pseudo-residuals):
        \[ r_{im} = y_i - p_{m-1}(x_i) \]
        \item Huấn luyện cây $h_m(x)$ để dự đoán $r_{im}$.
        \item Cập nhật mô hình trong không gian log-odds:
        \[ F_m(x) = F_{m-1}(x) + \eta \cdot \gamma_m(x) \]
        (Với $\gamma_m(x)$ là giá trị đầu ra của các lá trong cây $h_m$).
    \end{enumerate}
\end{enumerate}

\subsection{Ví dụ tính toán chi tiết}
Áp dụng quy trình trên vào bộ dữ liệu sau để dự đoán \textbf{Loves Troll 2} (Yes=1, No=0).
\begin{figure}[H]
    \centering
    \includegraphics[width=0.8\textwidth]{projects/GradientBoosting/images/classification_dataset.png}
    \caption{Bộ dữ liệu ví dụ cho bài toán phân loại.}
    \label{fig:classification_data}
\end{figure}

\subsubsection{Vòng lặp 1: Xây dựng cây đầu tiên}

\textbf{Bước 1: Khởi tạo ($F_0, p_0$)}
\begin{itemize}
    \item Xác suất trung bình: $\bar{p} = 4/6 \approx 0.67$.
    \item Log-odds ban đầu: $F_0 = \log(\frac{0.67}{1-0.67}) \approx 0.71$.
    \item Xác suất dự đoán ban đầu: $p_0 = \text{sigmoid}(0.71) \approx 0.67$.
\end{itemize}

\textbf{Bước 2: Tính Phần dư giả ($r_1$)}
\begin{table}[H]
\centering
\begin{tabular}{rcccr}
\toprule
\textbf{ID} & \textbf{Age} & \textbf{Loves Troll 2 (y)} & \textbf{$p_0$} & \textbf{Residual ($r_1$)} \\
\midrule
1 & 12 & 1 & 0.67 & 0.33 \\
2 & 87 & 1 & 0.67 & 0.33 \\
3 & 44 & 0 & 0.67 & -0.67 \\
4 & 19 & 0 & 0.67 & -0.67 \\
5 & 32 & 1 & 0.67 & 0.33 \\
6 & 14 & 1 & 0.67 & 0.33 \\
\bottomrule
\end{tabular}
\caption{Tính toán phần dư giả cho vòng lặp đầu tiên.}
\end{table}

\textbf{Bước 3 \& 4: Xây dựng Cây $h_1(x)$ và Cập nhật $F_1(x)$} \\
Tương tự như bài toán hồi quy, một cây mới sẽ được tạo để học các phần dư này. Mô hình log-odds $F_0$ sẽ được cập nhật thành $F_1$, dẫn đến các xác suất dự đoán mới $p_1$ được cải thiện.

\section{Thực hành với Python}
\label{subsec:gradient-boosting-advanced-python}
\subsection{Mã nguồn cho Bài toán Hồi quy}
\begin{minted}{python}
import pandas as pd
import numpy as np
import matplotlib.pyplot as plt
from sklearn.ensemble import GradientBoostingRegressor
from sklearn.preprocessing import OneHotEncoder
from sklearn.compose import ColumnTransformer
from sklearn.pipeline import Pipeline

# 1. Create DataFrame from data
data_reg = {
    'Height': [1.6, 1.6, 1.5, 1.8, 1.5, 1.4],
    'Favorite Color': ['Blue', 'Green', 'Blue', 'Red', 'Green', 'Blue'],
    'Gender': ['Male', 'Female', 'Female', 'Male', 'Male', 'Female'],
    'Weight': [88, 76, 56, 73, 77, 57]
}
df_reg = pd.DataFrame(data_reg)

# 2. Prepare data for the model
X = df_reg.drop('Weight', axis=1)
y = df_reg['Weight']

# Define preprocessing steps
categorical_features = ['Favorite Color', 'Gender']
one_hot_encoder = OneHotEncoder(handle_unknown='ignore')
preprocessor = ColumnTransformer(
    transformers=[('cat', one_hot_encoder, categorical_features)],
    remainder='passthrough' # Keep other columns (Height)
)

# 3. Build and Train the Model
# Use GradientBoostingRegressor with basic parameters
# n_estimators: number of trees (boosting rounds)
# max_depth=1: each tree is a simple stump
gbr = GradientBoostingRegressor(n_estimators=100, learning_rate=0.1, max_depth=1, random_state=42)

# Create a pipeline to combine preprocessing and the model
model_pipeline = Pipeline(steps=[('preprocessor', preprocessor), ('regressor', gbr)])

# Train the pipeline on the entire dataset
model_pipeline.fit(X, y)

# 4. Visualize the results
plt.style.use('seaborn-v0_8-whitegrid')
plt.figure(figsize=(10, 6))
# Plot the actual data points
plt.scatter(df_reg['Height'], y, color='red', s=100, edgecolor='k', alpha=0.7, label='Actual Data')

# To plot the prediction line, create points and sort by Height
X_test_sorted = X.sort_values(by='Height')
y_pred_sorted = model_pipeline.predict(X_test_sorted)
plt.plot(X_test_sorted['Height'], y_pred_sorted, color='blue', linewidth=3, label='GBM Prediction')

# Customize the plot
plt.title('Gradient Boosting Regression: Height vs Weight', fontsize=16)
plt.xlabel('Height (m)', fontsize=12)
plt.ylabel('Weight (kg)', fontsize=12)
plt.legend()
plt.show()
\end{minted}

\begin{figure}[H]
    \centering
    \includegraphics[width=0.9\textwidth]{projects/GradientBoosting/images/regression_plot.png}
    \caption{Kết quả trực quan hóa của mô hình hồi quy GBM.}
    \label{fig:regression_plot}
\end{figure}

\subsection{Mã nguồn cho Bài toán Phân loại}
\begin{minted}{python}
import pandas as pd
import numpy as np
import matplotlib.pyplot as plt
from sklearn.ensemble import GradientBoostingClassifier
from sklearn.preprocessing import LabelEncoder
from mlxtend.plotting import plot_decision_regions

# 1. Create DataFrame
data_cls = {
    'Likes Popcorn': ['Yes', 'Yes', 'No', 'Yes', 'No', 'No'],
    'Age': [12, 87, 44, 19, 32, 14],
    'Favorite Color': ['Blue', 'Green', 'Blue', 'Red', 'Green', 'Blue'],
    'Loves Troll 2': ['Yes', 'Yes', 'No', 'No', 'Yes', 'Yes']
}
df_cls = pd.DataFrame(data_cls)

# 2. Preprocessing (simplified for 2D visualization)
le_popcorn = LabelEncoder()
df_cls['Likes Popcorn'] = le_popcorn.fit_transform(df_cls['Likes Popcorn'])
y_encoder = LabelEncoder()
df_cls['Loves Troll 2'] = y_encoder.fit_transform(df_cls['Loves Troll 2'])

# Select 2 features for visualization
X_vis = df_cls[['Age', 'Likes Popcorn']].values
y_vis = df_cls['Loves Troll 2'].values

# 3. Build and Train the Model
gbc = GradientBoostingClassifier(n_estimators=100, learning_rate=0.1, max_depth=1, random_state=42)
gbc.fit(X_vis, y_vis)

# 4. Visualize the decision boundary
plt.style.use('seaborn-v0_8-whitegrid')
plt.figure(figsize=(10, 6))
plot_decision_regions(X_vis, y_vis, clf=gbc, legend=2)
plt.title('Gradient Boosting Classification: Decision Boundary', fontsize=16)
plt.xlabel('Age', fontsize=12)
plt.ylabel('Likes Popcorn (0: No, 1: Yes)', fontsize=12)
plt.show()
\end{minted}

\begin{figure}[H]
    \centering
    \includegraphics[width=0.9\textwidth]{projects/GradientBoosting/images/classification_plot.png}
    \caption{Đường biên quyết định của mô hình phân loại GBM.}
    \label{fig:classification_plot}
\end{figure}

\section{Tinh chỉnh siêu tham số để đạt hiệu suất tối ưu}
\begin{itemize}
    \item \textbf{\texttt{n\_estimators}}: Số lượng cây trong chuỗi.
    \item \textbf{\texttt{learning\_rate} ($\eta$)}: Tốc độ học. Có một sự đánh đổi quan trọng giữa \texttt{learning\_rate} và \texttt{n\_estimators}.
    \item \textbf{\texttt{max\_depth}}: Độ sâu tối đa của mỗi cây. Với GBM, chúng ta thường ưu tiên các cây nông (1 đến 5).
    \item \textbf{\texttt{subsample}}: Tỷ lệ mẫu dữ liệu được sử dụng để huấn luyện mỗi cây (Stochastic Gradient Boosting).
\end{itemize}

\section{So sánh với các thuật toán Ensemble khác}
\begin{table}[H]
\centering
\caption{So sánh Gradient Boosting, Random Forest, và AdaBoost.}
\begin{tabular}{l|p{3.5cm}|p{3.5cm}|p{3.5cm}}
\toprule
\textbf{Tiêu chí} & \textbf{Gradient Boosting} & \textbf{Random Forest}~\cite{breiman2001random} & \textbf{AdaBoost} \\
\midrule
\textbf{Thứ tự} & Tuần tự & Song song & Tuần tự \\
\textbf{Cơ chế học} & Học trên phần dư (gradient) & Học trên mẫu bootstrap độc lập & Học trên trọng số của các điểm bị sai \\
\textbf{Mục tiêu} & Giảm bias trước, sau đó giảm variance & Chủ yếu giảm variance & Chủ yếu giảm bias \\
\textbf{Độ sâu cây} & Cây nông & Cây sâu & Cây rất nông (stumps) \\
\bottomrule
\end{tabular}
\label{tab:comparison}
\end{table}

\section{Tổng kết}
Gradient Boosting không chỉ là một thuật toán; nó là một framework mạnh mẽ và có tính tổng quát cao~\cite{friedman2001greedy}. Bằng cách hiểu rõ cơ chế học tập theo gradient, chúng ta không chỉ nắm vững một công cụ mạnh mẽ mà còn có một nền tảng vững chắc để tiếp cận các thuật toán "hậu duệ" của nó như XGBoost~\cite{chen2016xgboost}, LightGBM và CatBoost, những "gã khổng lồ" đang thống trị nhiều cuộc thi về khoa học dữ liệu hiện nay.

% Tạo danh mục tài liệu tham khảo từ file references.bib
\bibliographystyle{plain}
\bibliography{projects/GradientBoosting/references}
\clearpage
\setcounter{section}{0} % Reset section numbering

\begin{center}
    \Large\textbf{Khám phá Gradient Boosting: Nghệ thuật điêu khắc mô hình từ sai số}
\end{center}

\begin{center}
    \Large\textit{Vũ Thái Sơn}
\end{center}

\begin{center}
    \large "Lấy sai số làm kim chỉ nam": Lý thuyết đến thực hành Gradient Boosting
\end{center}

\section{Giới thiệu: Ngoài việc học từ sai lầm, hãy học đúng hướng!}
\label{sec:gradient-boosting-advanced}

Hãy tưởng tượng bạn là một nhà điêu khắc đang tạc một bức tượng từ một khối đá cẩm thạch thô. Bạn không thể tạo ra kiệt tác ngay trong nhát đục đầu tiên. Thay vào đó, bạn bắt đầu với những đường nét cơ bản, sau đó lùi lại, quan sát những điểm chưa hoàn hảo -- "sai số" so với hình ảnh trong tâm trí -- rồi cẩn thận đục đẽo để sửa chữa những sai số đó. Mỗi lớp đá được loại bỏ là một bước đưa tác phẩm đến gần hơn với sự hoàn hảo.

Đây là triết lý cốt lõi của các thuật toán \textbf{Boosting} trong \textbf{Học tập Tập thể (Ensemble Learning)}. Thay vì xây dựng một mô hình phức tạp duy nhất, chúng ta kết hợp sức mạnh của nhiều mô hình đơn giản (weak learners), và mỗi mô hình sau sẽ học từ sai lầm của các mô hình trước. AdaBoost, một thuật toán tiên phong, thực hiện điều này bằng cách tăng trọng số cho các điểm bị dự đoán sai, buộc mô hình sau phải "chú ý" hơn đến chúng.

Nhưng Gradient Boosting còn tiến một bước xa hơn. Nó không chỉ hỏi "Tôi đã sai ở đâu?", mà còn hỏi \textbf{"Để sửa sai, tôi nên đi theo hướng nào là nhanh nhất?"}. Câu trả lời nằm ở hai chữ \textbf{"Gradient Descent"}. Gradient Boosting áp dụng một cách thiên tài ý tưởng tối ưu hóa này vào không gian của các mô hình, tạo ra một nghệ sĩ bậc thầy có khả năng điêu khắc nên những mô hình dự đoán với độ chính xác đáng kinh ngạc.

\begin{figure}[H]
    \centering
    \includegraphics[width=0.8\textwidth]{projects/GradientBoosting/images/boosting.png}
    \caption{Đi theo hướng Gradient để tìm loss bé nhất.}
    \label{fig:boosting_method}
\end{figure}

Trong bài viết này, chúng ta sẽ cùng nhau thực hiện một hành trình chi tiết, từ việc "giải phẫu" lý thuyết và công thức toán học, đến việc áp dụng nó vào hai ví dụ kinh điển: hồi quy và phân loại, với các bước tính toán cụ thể, và cuối cùng là hiện thực hóa bằng mã nguồn Python.

\section{Trái tim của thuật toán: "Gradient" đến từ đâu?}
\label{subsec:gradient-boosting-advanced-gradient}

Để thực sự hiểu Gradient Boosting, chúng ta cần nắm vững trực giác đằng sau chữ "Gradient". Hãy tưởng tượng bạn đang đứng trên một sườn núi trong sương mù, và mục tiêu là đi xuống thung lũng (nơi thấp nhất).

\begin{itemize}
    \item \textbf{Vị trí của bạn:} Tương đương với \textbf{dự đoán hiện tại} của mô hình.
    \item \textbf{Độ cao:} Tương đương với \textbf{sai số} (được đo bằng hàm mất mát - Loss Function).
    \item \textbf{Thung lũng:} Tương đương với \textbf{mô hình hoàn hảo} (sai số bằng 0).
\end{itemize}

Trong sương mù, bạn không thấy thung lũng. Cách duy nhất là cảm nhận \textbf{độ dốc (gradient)} ngay dưới chân mình. Gradient cho bạn biết hướng dốc lên nhất. Để đi xuống, bạn chỉ cần bước một bước nhỏ theo hướng \textbf{ngược lại với gradient (negative gradient)}. Lặp lại quá trình này, bạn sẽ dần dần đi tới đáy thung lũng.

Gradient Boosting đã "mượn" ý tưởng này. Nó xem việc tối ưu mô hình như một hành trình đi xuống "thung lũng sai số". \textbf{Phần dư (residual)} mà chúng ta sẽ tính toán ở các bước tiếp theo, về mặt toán học, chính là một xấp xỉ của \textbf{hướng negative gradient} đó. Mỗi cây quyết định mới được thêm vào chính là một "bước đi" theo hướng giảm sai số nhanh nhất.

\begin{figure}[H]
    \centering
    \includegraphics[width=0.8\textwidth]{projects/GradientBoosting/images/valley.png}
    \caption{Đi theo hướng Gradient để tìm loss bé nhất.}
    \label{fig:gradient_valley}
\end{figure}

\section{Gradient Boosting cho Bài toán Hồi quy}
\label{subsec:gradient-boosting-advanced-regression}
\subsection{Ý tưởng chính và Công thức}
Ý tưởng là xây dựng một chuỗi các cây quyết định, trong đó mỗi cây sau sẽ cố gắng dự đoán \textbf{phần dư (residual)} của tất cả các cây trước đó cộng lại.

\begin{enumerate}
    \item \textbf{Khởi tạo ($F_0$):} Bắt đầu với một dự đoán không đổi, thường là giá trị trung bình của biến mục tiêu $y$.
    \[ F_0(x) = \text{mean}(y) \]
    
    \item \textbf{Lặp (với $m$ từ 1 đến M):}
    \begin{enumerate}
        \item Tính phần dư (pseudo-residuals):
        \[ r_{im} = y_i - F_{m-1}(x_i) \]
        
        \item Huấn luyện một cây quyết định hồi quy $h_m(x)$ mới để dự đoán các phần dư $r_{im}$.
        
        \item Cập nhật mô hình tổng hợp:
        \[ F_m(x) = F_{m-1}(x) + \eta \cdot h_m(x) \]
    \end{enumerate}
\end{enumerate}

\subsection{Ví dụ tính toán chi tiết}
Hãy áp dụng quy trình trên vào bộ dữ liệu sau để dự đoán \textbf{Weight (kg)}.

\begin{figure}[H]
    \centering
    \includegraphics[width=0.8\textwidth]{projects/GradientBoosting/images/regression_dataset.png}
    \caption{Bộ dữ liệu ví dụ cho bài toán hồi quy.}
    \label{fig:regression_data}
\end{figure}

\subsubsection{Vòng lặp 1: Xây dựng cây đầu tiên}

\textbf{Bước 1: Khởi tạo ($F_0$)}
\[ F_0 = \frac{88+76+56+73+77+57}{6} = 71.17 \]

\textbf{Bước 2: Tính Phần dư ($r_1$)}
\begin{table}[H]
\centering
\begin{tabular}{rcccr}
\toprule
\textbf{ID} & \textbf{Height (m)} & \textbf{Weight (y)} & \textbf{$F_0$} & \textbf{Residual ($r_1$)} \\
\midrule
1 & 1.6 & 88 & 71.17 & 16.83 \\
2 & 1.6 & 76 & 71.17 & 4.83 \\
3 & 1.5 & 56 & 71.17 & -15.17 \\
4 & 1.8 & 73 & 71.17 & 1.83 \\
5 & 1.5 & 77 & 71.17 & 5.83 \\
6 & 1.4 & 57 & 71.17 & -14.17 \\
\bottomrule
\end{tabular}
\caption{Tính toán phần dư cho vòng lặp đầu tiên.}
\end{table}

\textbf{Bước 3: Xây dựng Cây $h_1(x)$}
Giả sử stump tốt nhất chia theo \textbf{Height $\leq$ 1.55}.
\begin{itemize}
    \item \textbf{Lá Trái (Height $\leq$ 1.55):} Gồm các điểm (ID 3, 5, 6).
    \[ \text{Leaf}_{\text{Left}} = \frac{-15.17 + 5.83 - 14.17}{3} \approx -7.84 \]
    \item \textbf{Lá Phải (Height > 1.55):} Gồm các điểm (ID 1, 2, 4).
    \[ \text{Leaf}_{\text{Right}} = \frac{16.83 + 4.83 + 1.83}{3} \approx 7.83 \]
\end{itemize}

\textbf{Bước 4: Cập nhật Mô hình ($F_1$)} \\
Giả sử tốc độ học $\eta = 0.1$.
\begin{table}[H]
\centering
\begin{tabular}{rccc}
\toprule
\textbf{ID} & \textbf{$F_0$} & \textbf{$h_1(x_i)$} & \textbf{$F_1(x_i) = F_0 + \eta \cdot h_1$} \\
\midrule
1 & 71.17 & 7.83 & $71.17 + 0.1 \times 7.83 = 71.95$ \\
2 & 71.17 & 7.83 & $71.17 + 0.1 \times 7.83 = 71.95$ \\
3 & 71.17 & -7.84 & $71.17 + 0.1 \times (-7.84) = 70.39$ \\
4 & 71.17 & 7.83 & $71.17 + 0.1 \times 7.83 = 71.95$ \\
5 & 71.17 & -7.84 & $71.17 + 0.1 \times (-7.84) = 70.39$ \\
6 & 71.17 & -7.84 & $71.17 + 0.1 \times (-7.84) = 70.39$ \\
\bottomrule
\end{tabular}
\caption{Cập nhật dự đoán sau vòng lặp đầu tiên.}
\end{table}

\subsubsection{Vòng lặp 2: Xây dựng cây thứ hai}
\textbf{Bước 5: Tính Phần dư mới ($r_2$)}
\begin{table}[H]
\centering
\begin{tabular}{rccc}
\toprule
\textbf{ID} & \textbf{Weight (y)} & \textbf{$F_1(x_i)$} & \textbf{Residual ($r_2 = y - F_1$)} \\
\midrule
1 & 88 & 71.95 & 16.05 \\
2 & 76 & 71.95 & 4.05 \\
3 & 56 & 70.39 & -14.39 \\
4 & 73 & 71.95 & 1.05 \\
5 & 77 & 70.39 & 6.61 \\
6 & 57 & 70.39 & -13.39 \\
\bottomrule
\end{tabular}
\caption{Tính toán phần dư cho vòng lặp thứ hai.}
\end{table}

\textbf{Bước 6: Xây dựng cây $h_2(x)$ và cập nhật $F_2(x)$} \\
Một cây quyết định mới, $h_2(x)$, sẽ được huấn luyện để dự đoán cột \textbf{Residual ($r_2$)}. Sau đó, mô hình sẽ được cập nhật một lần nữa: $F_2(x) = F_1(x) + \eta \cdot h_2(x)$.

\section{Gradient Boosting cho Bài toán Phân loại}
\label{subsec:gradient-boosting-advanced-classification}
\subsection{Ý tưởng chính và Công thức}
Thuật toán làm việc với xác suất và tối ưu trên không gian log-odds.

\begin{enumerate}
    \item \textbf{Khởi tạo ($F_0$):} Bắt đầu với một dự đoán log-odds không đổi.
    \[ F_0(x) = \log\left(\frac{p}{1-p}\right) \quad \text{với } p = \text{mean}(y) \]
    
    \item \textbf{Lặp (với $m$ từ 1 đến M):}
    \begin{enumerate}
        \item Chuyển log-odds $F_{m-1}(x)$ thành xác suất $p_{m-1}(x)$ bằng hàm Sigmoid.
        \item Tính phần dư giả (pseudo-residuals):
        \[ r_{im} = y_i - p_{m-1}(x_i) \]
        \item Huấn luyện cây $h_m(x)$ để dự đoán $r_{im}$.
        \item Cập nhật mô hình trong không gian log-odds:
        \[ F_m(x) = F_{m-1}(x) + \eta \cdot \gamma_m(x) \]
        (Với $\gamma_m(x)$ là giá trị đầu ra của các lá trong cây $h_m$).
    \end{enumerate}
\end{enumerate}

\subsection{Ví dụ tính toán chi tiết}
Áp dụng quy trình trên vào bộ dữ liệu sau để dự đoán \textbf{Loves Troll 2} (Yes=1, No=0).
\begin{figure}[H]
    \centering
    \includegraphics[width=0.8\textwidth]{projects/GradientBoosting/images/classification_dataset.png}
    \caption{Bộ dữ liệu ví dụ cho bài toán phân loại.}
    \label{fig:classification_data}
\end{figure}

\subsubsection{Vòng lặp 1: Xây dựng cây đầu tiên}

\textbf{Bước 1: Khởi tạo ($F_0, p_0$)}
\begin{itemize}
    \item Xác suất trung bình: $\bar{p} = 4/6 \approx 0.67$.
    \item Log-odds ban đầu: $F_0 = \log(\frac{0.67}{1-0.67}) \approx 0.71$.
    \item Xác suất dự đoán ban đầu: $p_0 = \text{sigmoid}(0.71) \approx 0.67$.
\end{itemize}

\textbf{Bước 2: Tính Phần dư giả ($r_1$)}
\begin{table}[H]
\centering
\begin{tabular}{rcccr}
\toprule
\textbf{ID} & \textbf{Age} & \textbf{Loves Troll 2 (y)} & \textbf{$p_0$} & \textbf{Residual ($r_1$)} \\
\midrule
1 & 12 & 1 & 0.67 & 0.33 \\
2 & 87 & 1 & 0.67 & 0.33 \\
3 & 44 & 0 & 0.67 & -0.67 \\
4 & 19 & 0 & 0.67 & -0.67 \\
5 & 32 & 1 & 0.67 & 0.33 \\
6 & 14 & 1 & 0.67 & 0.33 \\
\bottomrule
\end{tabular}
\caption{Tính toán phần dư giả cho vòng lặp đầu tiên.}
\end{table}

\textbf{Bước 3 \& 4: Xây dựng Cây $h_1(x)$ và Cập nhật $F_1(x)$} \\
Tương tự như bài toán hồi quy, một cây mới sẽ được tạo để học các phần dư này. Mô hình log-odds $F_0$ sẽ được cập nhật thành $F_1$, dẫn đến các xác suất dự đoán mới $p_1$ được cải thiện.

\section{Thực hành với Python}
\label{subsec:gradient-boosting-advanced-python}
\subsection{Mã nguồn cho Bài toán Hồi quy}
\begin{minted}{python}
import pandas as pd
import numpy as np
import matplotlib.pyplot as plt
from sklearn.ensemble import GradientBoostingRegressor
from sklearn.preprocessing import OneHotEncoder
from sklearn.compose import ColumnTransformer
from sklearn.pipeline import Pipeline

# 1. Create DataFrame from data
data_reg = {
    'Height': [1.6, 1.6, 1.5, 1.8, 1.5, 1.4],
    'Favorite Color': ['Blue', 'Green', 'Blue', 'Red', 'Green', 'Blue'],
    'Gender': ['Male', 'Female', 'Female', 'Male', 'Male', 'Female'],
    'Weight': [88, 76, 56, 73, 77, 57]
}
df_reg = pd.DataFrame(data_reg)

# 2. Prepare data for the model
X = df_reg.drop('Weight', axis=1)
y = df_reg['Weight']

# Define preprocessing steps
categorical_features = ['Favorite Color', 'Gender']
one_hot_encoder = OneHotEncoder(handle_unknown='ignore')
preprocessor = ColumnTransformer(
    transformers=[('cat', one_hot_encoder, categorical_features)],
    remainder='passthrough' # Keep other columns (Height)
)

# 3. Build and Train the Model
# Use GradientBoostingRegressor with basic parameters
# n_estimators: number of trees (boosting rounds)
# max_depth=1: each tree is a simple stump
gbr = GradientBoostingRegressor(n_estimators=100, learning_rate=0.1, max_depth=1, random_state=42)

# Create a pipeline to combine preprocessing and the model
model_pipeline = Pipeline(steps=[('preprocessor', preprocessor), ('regressor', gbr)])

# Train the pipeline on the entire dataset
model_pipeline.fit(X, y)

# 4. Visualize the results
plt.style.use('seaborn-v0_8-whitegrid')
plt.figure(figsize=(10, 6))
# Plot the actual data points
plt.scatter(df_reg['Height'], y, color='red', s=100, edgecolor='k', alpha=0.7, label='Actual Data')

# To plot the prediction line, create points and sort by Height
X_test_sorted = X.sort_values(by='Height')
y_pred_sorted = model_pipeline.predict(X_test_sorted)
plt.plot(X_test_sorted['Height'], y_pred_sorted, color='blue', linewidth=3, label='GBM Prediction')

# Customize the plot
plt.title('Gradient Boosting Regression: Height vs Weight', fontsize=16)
plt.xlabel('Height (m)', fontsize=12)
plt.ylabel('Weight (kg)', fontsize=12)
plt.legend()
plt.show()
\end{minted}

\begin{figure}[H]
    \centering
    \includegraphics[width=0.9\textwidth]{projects/GradientBoosting/images/regression_plot.png}
    \caption{Kết quả trực quan hóa của mô hình hồi quy GBM.}
    \label{fig:regression_plot}
\end{figure}

\subsection{Mã nguồn cho Bài toán Phân loại}
\begin{minted}{python}
import pandas as pd
import numpy as np
import matplotlib.pyplot as plt
from sklearn.ensemble import GradientBoostingClassifier
from sklearn.preprocessing import LabelEncoder
from mlxtend.plotting import plot_decision_regions

# 1. Create DataFrame
data_cls = {
    'Likes Popcorn': ['Yes', 'Yes', 'No', 'Yes', 'No', 'No'],
    'Age': [12, 87, 44, 19, 32, 14],
    'Favorite Color': ['Blue', 'Green', 'Blue', 'Red', 'Green', 'Blue'],
    'Loves Troll 2': ['Yes', 'Yes', 'No', 'No', 'Yes', 'Yes']
}
df_cls = pd.DataFrame(data_cls)

# 2. Preprocessing (simplified for 2D visualization)
le_popcorn = LabelEncoder()
df_cls['Likes Popcorn'] = le_popcorn.fit_transform(df_cls['Likes Popcorn'])
y_encoder = LabelEncoder()
df_cls['Loves Troll 2'] = y_encoder.fit_transform(df_cls['Loves Troll 2'])

# Select 2 features for visualization
X_vis = df_cls[['Age', 'Likes Popcorn']].values
y_vis = df_cls['Loves Troll 2'].values

# 3. Build and Train the Model
gbc = GradientBoostingClassifier(n_estimators=100, learning_rate=0.1, max_depth=1, random_state=42)
gbc.fit(X_vis, y_vis)

# 4. Visualize the decision boundary
plt.style.use('seaborn-v0_8-whitegrid')
plt.figure(figsize=(10, 6))
plot_decision_regions(X_vis, y_vis, clf=gbc, legend=2)
plt.title('Gradient Boosting Classification: Decision Boundary', fontsize=16)
plt.xlabel('Age', fontsize=12)
plt.ylabel('Likes Popcorn (0: No, 1: Yes)', fontsize=12)
plt.show()
\end{minted}

\begin{figure}[H]
    \centering
    \includegraphics[width=0.9\textwidth]{projects/GradientBoosting/images/classification_plot.png}
    \caption{Đường biên quyết định của mô hình phân loại GBM.}
    \label{fig:classification_plot}
\end{figure}

\section{Tinh chỉnh siêu tham số để đạt hiệu suất tối ưu}
\begin{itemize}
    \item \textbf{\texttt{n\_estimators}}: Số lượng cây trong chuỗi.
    \item \textbf{\texttt{learning\_rate} ($\eta$)}: Tốc độ học. Có một sự đánh đổi quan trọng giữa \texttt{learning\_rate} và \texttt{n\_estimators}.
    \item \textbf{\texttt{max\_depth}}: Độ sâu tối đa của mỗi cây. Với GBM, chúng ta thường ưu tiên các cây nông (1 đến 5).
    \item \textbf{\texttt{subsample}}: Tỷ lệ mẫu dữ liệu được sử dụng để huấn luyện mỗi cây (Stochastic Gradient Boosting).
\end{itemize}

\section{So sánh với các thuật toán Ensemble khác}
\begin{table}[H]
\centering
\caption{So sánh Gradient Boosting, Random Forest, và AdaBoost.}
\begin{tabular}{l|p{3.5cm}|p{3.5cm}|p{3.5cm}}
\toprule
\textbf{Tiêu chí} & \textbf{Gradient Boosting} & \textbf{Random Forest}~\cite{breiman2001random} & \textbf{AdaBoost} \\
\midrule
\textbf{Thứ tự} & Tuần tự & Song song & Tuần tự \\
\textbf{Cơ chế học} & Học trên phần dư (gradient) & Học trên mẫu bootstrap độc lập & Học trên trọng số của các điểm bị sai \\
\textbf{Mục tiêu} & Giảm bias trước, sau đó giảm variance & Chủ yếu giảm variance & Chủ yếu giảm bias \\
\textbf{Độ sâu cây} & Cây nông & Cây sâu & Cây rất nông (stumps) \\
\bottomrule
\end{tabular}
\label{tab:comparison}
\end{table}

\section{Tổng kết}
Gradient Boosting không chỉ là một thuật toán; nó là một framework mạnh mẽ và có tính tổng quát cao~\cite{friedman2001greedy}. Bằng cách hiểu rõ cơ chế học tập theo gradient, chúng ta không chỉ nắm vững một công cụ mạnh mẽ mà còn có một nền tảng vững chắc để tiếp cận các thuật toán "hậu duệ" của nó như XGBoost~\cite{chen2016xgboost}, LightGBM và CatBoost, những "gã khổng lồ" đang thống trị nhiều cuộc thi về khoa học dữ liệu hiện nay.

% Tạo danh mục tài liệu tham khảo từ file references.bib
\bibliographystyle{plain}
\bibliography{projects/GradientBoosting/references}
\clearpage
\setcounter{section}{0} % Reset section numbering

\begin{center}    
    \Large\textbf{Khám phá thuật toán Decision Tree: Từ lý thuyết đến các phương pháp nâng cấp}
\end{center}

\begin{center}
    \Large\textit{Vũ Thái Sơn}
\end{center}

\begin{center}
\large Hành trình tìm hiểu thuật toán "hỏi hàng xóm" trong học máy
\end{center}

\section{Giới thiệu: Khám phá Decision Tree}
Decision Tree (Cây quyết định) là một trong những thuật toán nền tảng và dễ hiểu nhất trong lĩnh vực học máy. Hình dung đơn giản: Decision Tree mô phỏng quá trình con người đặt câu hỏi và đưa ra quyết định dựa trên các thuộc tính quan sát được. Khi học về cây quyết định, bạn sẽ thấy cách dữ liệu được phân loại từng bước, gần gũi như cách chúng ta suy nghĩ – từ tổng quát đến chi tiết, đến khi ra quyết định cuối cùng~\cite{al_2025_a}.

\section{Quy tắc hoạt động của Decision Tree}
\subsection{Cấu trúc cây quyết định}
Một cây quyết định bao gồm:
\begin{itemize}
    \item \textbf{Nút gốc (Root Node):} Nút đầu tiên chứa toàn bộ dữ liệu.
    \item \textbf{Nút trong (Internal Node):} Đại diện cho các điều kiện hoặc câu hỏi trên từng thuộc tính.
    \item \textbf{Nút lá (Leaf Node):} Cho ta kết quả cuối cùng sau các phép phân tách.
\end{itemize}
\begin{figure}[H]
    \centering
    \includegraphics[width=0.7\textwidth]{projects/DTree-DD/image/decision-tree-structure.png}
    \caption{Cấu trúc một cây quyết định đơn giản.}
    \label{fig:dt-structure}
\end{figure}
\subsection{Cách thuật toán chọn nhánh (splitting)}
Mục đích chia nhánh là làm các nút lá càng “thuần khiết” càng tốt, tức dữ liệu trong mỗi nút có cùng một lớp (hoặc giá trị dự đoán) nhiều nhất có thể.

\textbf{Các phép đo mức độ thuần khiết:}
\begin{itemize}
    \item \textbf{Gini impurity:} Đo lường xác suất một mẫu bị gán nhãn sai nếu chọn ngẫu nhiên.
        \[
        Gini(D) = 1 - \sum_{i=1}^{k} p_i^2
        \]
    \item \textbf{Entropy:} Đo lường tính ngẫu nhiên trong dữ liệu.
        \[
        H(X) = -\sum_{i=1}^c p_i \log_2 p_i
        \]
\end{itemize}
Thông qua từng bước tách, thuật toán sẽ chọn thuộc tính giúp giảm độ không thuần khiết nhiều nhất.

\section{Ứng dụng Decision Tree: Phân loại thực tế}
Để hiểu rõ hơn về cách Decision Tree hoạt động, chúng ta hãy áp dụng nó vào một ví dụ thực tế. Giả sử chúng ta có một bộ dữ liệu về hành vi của khách hàng tiềm năng và muốn dự đoán liệu họ có đăng ký khóa học hay không. Dưới đây là bộ dữ liệu chi tiết hơn được sử dụng để xây dựng cây quyết định trong phần thực hành Python.

\begin{table}[H]
\centering
\begin{tabular}{|c|c|c|c|c|}
\hline
\textbf{ID} & \textbf{Giờ xem} & \textbf{Số click} & \textbf{Webinar} & \textbf{Kết quả} \\
\hline
1 & 1.5 & 3 & Có & Có \\
2 & 2.0 & 1 & Không & Không \\
3 & 0.5 & 5 & Có & Có \\
4 & 3.0 & 2 & Không & Không \\
5 & 1.0 & 4 & Có & Có \\
6 & 2.5 & 1 & Không & Không \\
7 & 1.8 & 4 & Có & Có \\
8 & 2.4 & 2 & Không & Có \\
9 & 2.9 & 3 & Có & Không \\
10 & 1.2 & 2 & Có & Không \\
11 & 0.8 & 3 & Không & Không \\
12 & 2.6 & 5 & Có & Có \\
13 & 0.7 & 2 & Không & Không \\
\hline
\end{tabular}
\caption{Bộ dữ liệu khách hàng mở rộng được dùng trong ví dụ này}
\label{tab:extended_data}
\end{table}

\subsection{Diễn giải quá trình phân tách của cây quyết định}

\textbf{Bước 1: Tính Gini Impurity của Nút Gốc}

Nút gốc đại diện cho toàn bộ 13 mẫu dữ liệu. Chúng ta đếm được 7 khách hàng có kết quả \textbf{Không} (lớp 0) và 6 khách hàng có kết quả \textbf{Có} (lớp 1).

Chúng ta sử dụng công thức Gini Impurity để đo lường độ không thuần khiết của nút này.

\[
Gini_{\text{root}} = 1 - \left( \left(\frac{7}{13}\right)^2 + \left(\frac{6}{13}\right)^2 \right) = 1 - \left( \frac{49}{169} + \frac{36}{169} \right) = 1 - \frac{85}{169} \approx 0.497
\]

Giá trị $Gini \approx 0.497$ khớp với thông tin hiển thị tại nút gốc trên hình ảnh cây quyết định.

\textbf{Bước 2: Chọn thuộc tính phân tách tối ưu: $\text{so\_click} \leq 3.5$}

Thuật toán sẽ đánh giá tất cả các thuộc tính (\text{gio\_xem}, \text{so\_click}, \text{webinar}) để tìm ngưỡng phân tách giúp giảm Gini Impurity nhiều nhất.

Cây quyết định đã chọn ngưỡng $\text{so\_click} \leq 3.5$ là tối ưu nhất.

\begin{itemize}
    \item \textbf{Nhánh bên trái (điều kiện True):}
    \begin{itemize}
        \item Bao gồm 9 mẫu dữ liệu có $\text{so\_click} \leq 3.5$.
        \item Phân bố: 7 mẫu thuộc lớp \textbf{Không} và 2 mẫu thuộc lớp \textbf{Có}.
        \item Gini Impurity của nhánh này được tính là:
        \[
        Gini_{\text{left}} = 1 - \left( \left(\frac{7}{9}\right)^2 + \left(\frac{2}{9}\right)^2 \right) = 1 - \left( \frac{49}{81} + \frac{4}{81} \right) = 1 - \frac{53}{81} \approx 0.346
        \]
    \end{itemize}
    \item \textbf{Nhánh bên phải (điều kiện False):}
    \begin{itemize}
        \item Bao gồm 4 mẫu dữ liệu có $\text{so\_click} > 3.5$.
        \item Phân bố: 0 mẫu thuộc lớp \textbf{Không} và 4 mẫu thuộc lớp \textbf{Có}.
        \item Gini Impurity của nhánh này là:
        \[
        Gini_{\text{right}} = 1 - \left( \left(\frac{0}{4}\right)^2 + \left(\frac{4}{4}\right)^2 \right) = 1 - (0+1) = 0
        \]
    \end{itemize}
\end{itemize}

\textbf{Bước 3: Tiếp tục phân tách}

Vì nhánh bên trái ($Gini \approx 0.346$) vẫn chưa thuần khiết, quá trình phân tách tiếp tục. Thuật toán tìm ra điểm phân tách tối ưu tiếp theo là $\text{gio\_xem} \leq 1.35$. Quá trình này sẽ lặp lại cho đến khi các nút lá đạt Gini Impurity bằng 0 hoặc đạt đến giới hạn độ sâu của cây (trong trường hợp này là $\mathtt{max\_depth}=3$ như đã thiết lập trong mã nguồn).

\section{Thực hành với Python}
Minh hoạ xây dựng cây quyết định bằng scikit-learn và trực quan hoá.

\begin{lstlisting}[language=Python, caption={Xây dựng Decision Tree với scikit-learn}]
import pandas as pd
from sklearn.tree import DecisionTreeClassifier, plot_tree
import matplotlib.pyplot as plt

data = {
    'gio_xem': [1.5, 2.0, 0.5, 3.0, 1.0, 2.5, 1.8, 2.4, 2.9, 1.2, 0.8, 2.6, 0.7],
    'so_click': [3, 1, 5, 2, 4, 1, 4, 2, 3, 2, 3, 5, 2],
    'webinar': ['Co', 'Khong', 'Co', 'Khong', 'Co', 'Khong', 'Co', 'Khong', 'Co', 'Co', 'Khong', 'Co', 'Khong'],
    'ket_qua': ['Co', 'Khong', 'Co', 'Khong', 'Co', 'Khong', 'Co', 'Co', 'Khong', 'Khong', 'Khong', 'Co', 'Khong']
}
df = pd.DataFrame(data)
df['webinar'] = df['webinar'].map({'Co': 1, 'Khong': 0})
df['ket_qua'] = df['ket_qua'].map({'Co': 1, 'Khong': 0})
X = df[['gio_xem', 'so_click', 'webinar']]
y = df['ket_qua']

clf = DecisionTreeClassifier(max_depth=3, criterion='gini', random_state=42)
clf.fit(X, y)

plt.figure(figsize=(14,8))
plot_tree(
    clf,
    feature_names=['gio_xem', 'so_click', 'webinar'],
    class_names=['Khong','Co'],
    filled=True,
    rounded=True,
    fontsize=10
)
plt.title("Decision tree: customer classification with many branches", fontsize=14)
plt.tight_layout()
plt.show()
\end{lstlisting}

\begin{figure}[H]
    \centering
    \includegraphics[width=0.8\textwidth]{projects/DTree-DD/image/dt-python-example.png}
    \caption{Trực quan cây quyết định trên dữ liệu khách hàng.}
    \label{fig:dt-python-example}
\end{figure}

\section{Ưu điểm và hạn chế của Decision Tree}
\subsection{Ưu điểm}
\begin{itemize}
    \item Dễ hiểu, dễ diễn giải: Sơ đồ cây trực quan hoá logic ra quyết định.
    \item Xử lý được cả dữ liệu số và phân loại mà không cần nhiều bước tiền xử lý.
    \item Áp dụng cho cả phân loại lẫn hồi quy.
\end{itemize}

\subsection{Hạn chế}
\begin{itemize}
    \item Dễ bị overfitting với dữ liệu nhiễu hoặc nhỏ – mô hình quá khớp với dữ liệu huấn luyện.
    \item Ít ổn định: Chỉ cần thay đổi nhỏ trong dữ liệu có thể tạo ra cây rất khác nhau.
\end{itemize}

\begin{figure}[H]
    \centering
    \includegraphics[width=0.7\textwidth]{projects/DTree-DD/image/overfitting.png}
    \caption{Minh hoạ vấn đề Overfitting: cây quá phức tạp học thuộc dữ liệu huấn luyện.}
    \label{fig:overfitting}
\end{figure}

\section{Cách khắc phục overfitting}
Để kiểm soát độ phức tạp cây, có hai kỹ thuật chính:
\begin{itemize}
    \item \textbf{Pre-Pruning:} Đặt giới hạn cho cây khi xây dựng bằng cách dùng các tham số như $\mathtt{max\_depth}$, $\mathtt{min\_samples\_leaf}$, $\mathtt{min\_samples\_split}$.
    \item \textbf{Post-Pruning:} Xây cây hoàn chỉnh rồi cắt bớt các nhánh không cần thiết dựa trên kiểm thử.
\end{itemize}
\textbf{Ví dụ các tham số khi huấn luyện:}
\begin{lstlisting}[language=Python]
clf = DecisionTreeClassifier(max_depth=3, min_samples_leaf=2)
\end{lstlisting}

\begin{figure}[H]
    \centering
    \includegraphics[width=0.65\textwidth]{projects/DTree-DD/image/pruning-illustration.png}
    \caption{Hình minh hoạ cây quyết định trước và sau khi cắt tỉa.}
    \label{fig:pruning}
\end{figure}

\section{Ứng dụng đa lĩnh vực}
Decision Tree được sử dụng rộng rãi~\cite{al_2025_a}:
\begin{enumerate}
    \item \textbf{Y tế:} Hỗ trợ chẩn đoán bệnh từ các thuộc tính lâm sàng.
    \item \textbf{Tài chính ngân hàng:} Đánh giá tín dụng, dự đoán khả năng vỡ nợ.
    \item \textbf{Giáo dục:} Phân loại học viên tiềm năng, dự báo kết quả học tập.
\end{enumerate}

\section{So sánh Decision Tree với các thuật toán phân loại khác}
\begin{table}[H]
\centering
\begin{tabular}{|l|c|c|c|}
\hline
\textbf{Thuật toán} & \textbf{Mức độ trực quan} & \textbf{Hiệu năng với dữ liệu lớn} & \textbf{Độ dễ dùng} \\
\hline
Decision Tree & Rất cao & TB & Rất dễ \\
KNN & TB & Thấp & TB \\
Logistic Regression & TB & Cao & TB \\
\hline
\end{tabular}
\caption{So sánh Decision Tree với KNN, Logistic Regression}
\end{table}
\textbf{Nhận xét:}
- Decision Tree có ưu điểm lớn về tính trực quan và dễ hiểu nhất cho người mới học.

\section{Các phương pháp nâng cấp: Ensemble cơ bản}
\subsection{Giới thiệu Random Forest}
Random Forest kết hợp nhiều cây quyết định nhỏ (cây con), mỗi cây huấn luyện trên một tập dữ liệu con (bagging). Khi dự đoán, các cây cùng “bỏ phiếu”, kết quả là đa số phiếu hoặc trung bình. Phương pháp này giúp giảm overfitting, tăng tính ổn định \& chính xác so với cây đơn lẻ~\cite{breiman2001random}.

\begin{figure}[H]
    \centering
    \includegraphics[width=0.7\textwidth]{projects/DTree-DD/image/randomforest-illustration.png}
    \caption{Minh hoạ Random Forest và quá trình “bỏ phiếu” tổng hợp kết quả.}
\end{figure}

\section{Giải thích mô hình: Tại sao Decision Tree “interpretable”}
Cấu trúc Decision Tree cho phép bạn đọc dễ dàng lý giải vì sao một dự đoán được đưa ra (nhờ các bộ quy tắc từ gốc tới lá). Tính minh bạch này đặc biệt quan trọng trong các ngành tài chính, y tế, nơi cần giải thích rõ quyết định cho người dùng. Phương pháp hiện đại như SHAP, LIME còn cho phép giải thích cả các mô hình phức tạp hơn dựa trên nguyên lý Decision Tree.

\section{Tổng kết}
Decision Tree là bước khởi đầu lý tưởng cho người học AI. Bạn dễ dàng vừa học lý thuyết, vừa thực hành. Qua các ví dụ, bạn có thể ứng dụng vào nhiều lĩnh vực. Nắm vững Decision Tree sẽ là nền móng vững chắc để tiếp cận các mô hình phát triển hơn như Random Forest\cite{breiman2001random}, Gradient Boosting\cite{chen2016xgboost}.

\nocite{*}
\bibliographystyle{IEEEtran}
\bibliography{projects/DTree-DD/references} % Đường dẫn tới refs.bib
\clearpage
\setcounter{section}{0} % Reset section numbering

\begin{center}
    \Large\textbf{Khám phá Gradient Boosting: Nghệ thuật điêu khắc mô hình từ sai số}
\end{center}

\begin{center}
    \Large\textit{Vũ Thái Sơn}
\end{center}

\begin{center}
    \large "Lấy sai số làm kim chỉ nam": Lý thuyết đến thực hành Gradient Boosting
\end{center}

\section{Giới thiệu: Ngoài việc học từ sai lầm, hãy học đúng hướng!}
\label{sec:gradient-boosting-advanced}

Hãy tưởng tượng bạn là một nhà điêu khắc đang tạc một bức tượng từ một khối đá cẩm thạch thô. Bạn không thể tạo ra kiệt tác ngay trong nhát đục đầu tiên. Thay vào đó, bạn bắt đầu với những đường nét cơ bản, sau đó lùi lại, quan sát những điểm chưa hoàn hảo -- "sai số" so với hình ảnh trong tâm trí -- rồi cẩn thận đục đẽo để sửa chữa những sai số đó. Mỗi lớp đá được loại bỏ là một bước đưa tác phẩm đến gần hơn với sự hoàn hảo.

Đây là triết lý cốt lõi của các thuật toán \textbf{Boosting} trong \textbf{Học tập Tập thể (Ensemble Learning)}. Thay vì xây dựng một mô hình phức tạp duy nhất, chúng ta kết hợp sức mạnh của nhiều mô hình đơn giản (weak learners), và mỗi mô hình sau sẽ học từ sai lầm của các mô hình trước. AdaBoost, một thuật toán tiên phong, thực hiện điều này bằng cách tăng trọng số cho các điểm bị dự đoán sai, buộc mô hình sau phải "chú ý" hơn đến chúng.

Nhưng Gradient Boosting còn tiến một bước xa hơn. Nó không chỉ hỏi "Tôi đã sai ở đâu?", mà còn hỏi \textbf{"Để sửa sai, tôi nên đi theo hướng nào là nhanh nhất?"}. Câu trả lời nằm ở hai chữ \textbf{"Gradient Descent"}. Gradient Boosting áp dụng một cách thiên tài ý tưởng tối ưu hóa này vào không gian của các mô hình, tạo ra một nghệ sĩ bậc thầy có khả năng điêu khắc nên những mô hình dự đoán với độ chính xác đáng kinh ngạc.

\begin{figure}[H]
    \centering
    \includegraphics[width=0.8\textwidth]{projects/GradientBoosting/images/boosting.png}
    \caption{Đi theo hướng Gradient để tìm loss bé nhất.}
    \label{fig:boosting_method}
\end{figure}

Trong bài viết này, chúng ta sẽ cùng nhau thực hiện một hành trình chi tiết, từ việc "giải phẫu" lý thuyết và công thức toán học, đến việc áp dụng nó vào hai ví dụ kinh điển: hồi quy và phân loại, với các bước tính toán cụ thể, và cuối cùng là hiện thực hóa bằng mã nguồn Python.

\section{Trái tim của thuật toán: "Gradient" đến từ đâu?}
\label{subsec:gradient-boosting-advanced-gradient}

Để thực sự hiểu Gradient Boosting, chúng ta cần nắm vững trực giác đằng sau chữ "Gradient". Hãy tưởng tượng bạn đang đứng trên một sườn núi trong sương mù, và mục tiêu là đi xuống thung lũng (nơi thấp nhất).

\begin{itemize}
    \item \textbf{Vị trí của bạn:} Tương đương với \textbf{dự đoán hiện tại} của mô hình.
    \item \textbf{Độ cao:} Tương đương với \textbf{sai số} (được đo bằng hàm mất mát - Loss Function).
    \item \textbf{Thung lũng:} Tương đương với \textbf{mô hình hoàn hảo} (sai số bằng 0).
\end{itemize}

Trong sương mù, bạn không thấy thung lũng. Cách duy nhất là cảm nhận \textbf{độ dốc (gradient)} ngay dưới chân mình. Gradient cho bạn biết hướng dốc lên nhất. Để đi xuống, bạn chỉ cần bước một bước nhỏ theo hướng \textbf{ngược lại với gradient (negative gradient)}. Lặp lại quá trình này, bạn sẽ dần dần đi tới đáy thung lũng.

Gradient Boosting đã "mượn" ý tưởng này. Nó xem việc tối ưu mô hình như một hành trình đi xuống "thung lũng sai số". \textbf{Phần dư (residual)} mà chúng ta sẽ tính toán ở các bước tiếp theo, về mặt toán học, chính là một xấp xỉ của \textbf{hướng negative gradient} đó. Mỗi cây quyết định mới được thêm vào chính là một "bước đi" theo hướng giảm sai số nhanh nhất.

\begin{figure}[H]
    \centering
    \includegraphics[width=0.8\textwidth]{projects/GradientBoosting/images/valley.png}
    \caption{Đi theo hướng Gradient để tìm loss bé nhất.}
    \label{fig:gradient_valley}
\end{figure}

\section{Gradient Boosting cho Bài toán Hồi quy}
\label{subsec:gradient-boosting-advanced-regression}
\subsection{Ý tưởng chính và Công thức}
Ý tưởng là xây dựng một chuỗi các cây quyết định, trong đó mỗi cây sau sẽ cố gắng dự đoán \textbf{phần dư (residual)} của tất cả các cây trước đó cộng lại.

\begin{enumerate}
    \item \textbf{Khởi tạo ($F_0$):} Bắt đầu với một dự đoán không đổi, thường là giá trị trung bình của biến mục tiêu $y$.
    \[ F_0(x) = \text{mean}(y) \]
    
    \item \textbf{Lặp (với $m$ từ 1 đến M):}
    \begin{enumerate}
        \item Tính phần dư (pseudo-residuals):
        \[ r_{im} = y_i - F_{m-1}(x_i) \]
        
        \item Huấn luyện một cây quyết định hồi quy $h_m(x)$ mới để dự đoán các phần dư $r_{im}$.
        
        \item Cập nhật mô hình tổng hợp:
        \[ F_m(x) = F_{m-1}(x) + \eta \cdot h_m(x) \]
    \end{enumerate}
\end{enumerate}

\subsection{Ví dụ tính toán chi tiết}
Hãy áp dụng quy trình trên vào bộ dữ liệu sau để dự đoán \textbf{Weight (kg)}.

\begin{figure}[H]
    \centering
    \includegraphics[width=0.8\textwidth]{projects/GradientBoosting/images/regression_dataset.png}
    \caption{Bộ dữ liệu ví dụ cho bài toán hồi quy.}
    \label{fig:regression_data}
\end{figure}

\subsubsection{Vòng lặp 1: Xây dựng cây đầu tiên}

\textbf{Bước 1: Khởi tạo ($F_0$)}
\[ F_0 = \frac{88+76+56+73+77+57}{6} = 71.17 \]

\textbf{Bước 2: Tính Phần dư ($r_1$)}
\begin{table}[H]
\centering
\begin{tabular}{rcccr}
\toprule
\textbf{ID} & \textbf{Height (m)} & \textbf{Weight (y)} & \textbf{$F_0$} & \textbf{Residual ($r_1$)} \\
\midrule
1 & 1.6 & 88 & 71.17 & 16.83 \\
2 & 1.6 & 76 & 71.17 & 4.83 \\
3 & 1.5 & 56 & 71.17 & -15.17 \\
4 & 1.8 & 73 & 71.17 & 1.83 \\
5 & 1.5 & 77 & 71.17 & 5.83 \\
6 & 1.4 & 57 & 71.17 & -14.17 \\
\bottomrule
\end{tabular}
\caption{Tính toán phần dư cho vòng lặp đầu tiên.}
\end{table}

\textbf{Bước 3: Xây dựng Cây $h_1(x)$}
Giả sử stump tốt nhất chia theo \textbf{Height $\leq$ 1.55}.
\begin{itemize}
    \item \textbf{Lá Trái (Height $\leq$ 1.55):} Gồm các điểm (ID 3, 5, 6).
    \[ \text{Leaf}_{\text{Left}} = \frac{-15.17 + 5.83 - 14.17}{3} \approx -7.84 \]
    \item \textbf{Lá Phải (Height > 1.55):} Gồm các điểm (ID 1, 2, 4).
    \[ \text{Leaf}_{\text{Right}} = \frac{16.83 + 4.83 + 1.83}{3} \approx 7.83 \]
\end{itemize}

\textbf{Bước 4: Cập nhật Mô hình ($F_1$)} \\
Giả sử tốc độ học $\eta = 0.1$.
\begin{table}[H]
\centering
\begin{tabular}{rccc}
\toprule
\textbf{ID} & \textbf{$F_0$} & \textbf{$h_1(x_i)$} & \textbf{$F_1(x_i) = F_0 + \eta \cdot h_1$} \\
\midrule
1 & 71.17 & 7.83 & $71.17 + 0.1 \times 7.83 = 71.95$ \\
2 & 71.17 & 7.83 & $71.17 + 0.1 \times 7.83 = 71.95$ \\
3 & 71.17 & -7.84 & $71.17 + 0.1 \times (-7.84) = 70.39$ \\
4 & 71.17 & 7.83 & $71.17 + 0.1 \times 7.83 = 71.95$ \\
5 & 71.17 & -7.84 & $71.17 + 0.1 \times (-7.84) = 70.39$ \\
6 & 71.17 & -7.84 & $71.17 + 0.1 \times (-7.84) = 70.39$ \\
\bottomrule
\end{tabular}
\caption{Cập nhật dự đoán sau vòng lặp đầu tiên.}
\end{table}

\subsubsection{Vòng lặp 2: Xây dựng cây thứ hai}
\textbf{Bước 5: Tính Phần dư mới ($r_2$)}
\begin{table}[H]
\centering
\begin{tabular}{rccc}
\toprule
\textbf{ID} & \textbf{Weight (y)} & \textbf{$F_1(x_i)$} & \textbf{Residual ($r_2 = y - F_1$)} \\
\midrule
1 & 88 & 71.95 & 16.05 \\
2 & 76 & 71.95 & 4.05 \\
3 & 56 & 70.39 & -14.39 \\
4 & 73 & 71.95 & 1.05 \\
5 & 77 & 70.39 & 6.61 \\
6 & 57 & 70.39 & -13.39 \\
\bottomrule
\end{tabular}
\caption{Tính toán phần dư cho vòng lặp thứ hai.}
\end{table}

\textbf{Bước 6: Xây dựng cây $h_2(x)$ và cập nhật $F_2(x)$} \\
Một cây quyết định mới, $h_2(x)$, sẽ được huấn luyện để dự đoán cột \textbf{Residual ($r_2$)}. Sau đó, mô hình sẽ được cập nhật một lần nữa: $F_2(x) = F_1(x) + \eta \cdot h_2(x)$.

\section{Gradient Boosting cho Bài toán Phân loại}
\label{subsec:gradient-boosting-advanced-classification}
\subsection{Ý tưởng chính và Công thức}
Thuật toán làm việc với xác suất và tối ưu trên không gian log-odds.

\begin{enumerate}
    \item \textbf{Khởi tạo ($F_0$):} Bắt đầu với một dự đoán log-odds không đổi.
    \[ F_0(x) = \log\left(\frac{p}{1-p}\right) \quad \text{với } p = \text{mean}(y) \]
    
    \item \textbf{Lặp (với $m$ từ 1 đến M):}
    \begin{enumerate}
        \item Chuyển log-odds $F_{m-1}(x)$ thành xác suất $p_{m-1}(x)$ bằng hàm Sigmoid.
        \item Tính phần dư giả (pseudo-residuals):
        \[ r_{im} = y_i - p_{m-1}(x_i) \]
        \item Huấn luyện cây $h_m(x)$ để dự đoán $r_{im}$.
        \item Cập nhật mô hình trong không gian log-odds:
        \[ F_m(x) = F_{m-1}(x) + \eta \cdot \gamma_m(x) \]
        (Với $\gamma_m(x)$ là giá trị đầu ra của các lá trong cây $h_m$).
    \end{enumerate}
\end{enumerate}

\subsection{Ví dụ tính toán chi tiết}
Áp dụng quy trình trên vào bộ dữ liệu sau để dự đoán \textbf{Loves Troll 2} (Yes=1, No=0).
\begin{figure}[H]
    \centering
    \includegraphics[width=0.8\textwidth]{projects/GradientBoosting/images/classification_dataset.png}
    \caption{Bộ dữ liệu ví dụ cho bài toán phân loại.}
    \label{fig:classification_data}
\end{figure}

\subsubsection{Vòng lặp 1: Xây dựng cây đầu tiên}

\textbf{Bước 1: Khởi tạo ($F_0, p_0$)}
\begin{itemize}
    \item Xác suất trung bình: $\bar{p} = 4/6 \approx 0.67$.
    \item Log-odds ban đầu: $F_0 = \log(\frac{0.67}{1-0.67}) \approx 0.71$.
    \item Xác suất dự đoán ban đầu: $p_0 = \text{sigmoid}(0.71) \approx 0.67$.
\end{itemize}

\textbf{Bước 2: Tính Phần dư giả ($r_1$)}
\begin{table}[H]
\centering
\begin{tabular}{rcccr}
\toprule
\textbf{ID} & \textbf{Age} & \textbf{Loves Troll 2 (y)} & \textbf{$p_0$} & \textbf{Residual ($r_1$)} \\
\midrule
1 & 12 & 1 & 0.67 & 0.33 \\
2 & 87 & 1 & 0.67 & 0.33 \\
3 & 44 & 0 & 0.67 & -0.67 \\
4 & 19 & 0 & 0.67 & -0.67 \\
5 & 32 & 1 & 0.67 & 0.33 \\
6 & 14 & 1 & 0.67 & 0.33 \\
\bottomrule
\end{tabular}
\caption{Tính toán phần dư giả cho vòng lặp đầu tiên.}
\end{table}

\textbf{Bước 3 \& 4: Xây dựng Cây $h_1(x)$ và Cập nhật $F_1(x)$} \\
Tương tự như bài toán hồi quy, một cây mới sẽ được tạo để học các phần dư này. Mô hình log-odds $F_0$ sẽ được cập nhật thành $F_1$, dẫn đến các xác suất dự đoán mới $p_1$ được cải thiện.

\section{Thực hành với Python}
\label{subsec:gradient-boosting-advanced-python}
\subsection{Mã nguồn cho Bài toán Hồi quy}
\begin{minted}{python}
import pandas as pd
import numpy as np
import matplotlib.pyplot as plt
from sklearn.ensemble import GradientBoostingRegressor
from sklearn.preprocessing import OneHotEncoder
from sklearn.compose import ColumnTransformer
from sklearn.pipeline import Pipeline

# 1. Create DataFrame from data
data_reg = {
    'Height': [1.6, 1.6, 1.5, 1.8, 1.5, 1.4],
    'Favorite Color': ['Blue', 'Green', 'Blue', 'Red', 'Green', 'Blue'],
    'Gender': ['Male', 'Female', 'Female', 'Male', 'Male', 'Female'],
    'Weight': [88, 76, 56, 73, 77, 57]
}
df_reg = pd.DataFrame(data_reg)

# 2. Prepare data for the model
X = df_reg.drop('Weight', axis=1)
y = df_reg['Weight']

# Define preprocessing steps
categorical_features = ['Favorite Color', 'Gender']
one_hot_encoder = OneHotEncoder(handle_unknown='ignore')
preprocessor = ColumnTransformer(
    transformers=[('cat', one_hot_encoder, categorical_features)],
    remainder='passthrough' # Keep other columns (Height)
)

# 3. Build and Train the Model
# Use GradientBoostingRegressor with basic parameters
# n_estimators: number of trees (boosting rounds)
# max_depth=1: each tree is a simple stump
gbr = GradientBoostingRegressor(n_estimators=100, learning_rate=0.1, max_depth=1, random_state=42)

# Create a pipeline to combine preprocessing and the model
model_pipeline = Pipeline(steps=[('preprocessor', preprocessor), ('regressor', gbr)])

# Train the pipeline on the entire dataset
model_pipeline.fit(X, y)

# 4. Visualize the results
plt.style.use('seaborn-v0_8-whitegrid')
plt.figure(figsize=(10, 6))
# Plot the actual data points
plt.scatter(df_reg['Height'], y, color='red', s=100, edgecolor='k', alpha=0.7, label='Actual Data')

# To plot the prediction line, create points and sort by Height
X_test_sorted = X.sort_values(by='Height')
y_pred_sorted = model_pipeline.predict(X_test_sorted)
plt.plot(X_test_sorted['Height'], y_pred_sorted, color='blue', linewidth=3, label='GBM Prediction')

# Customize the plot
plt.title('Gradient Boosting Regression: Height vs Weight', fontsize=16)
plt.xlabel('Height (m)', fontsize=12)
plt.ylabel('Weight (kg)', fontsize=12)
plt.legend()
plt.show()
\end{minted}

\begin{figure}[H]
    \centering
    \includegraphics[width=0.9\textwidth]{projects/GradientBoosting/images/regression_plot.png}
    \caption{Kết quả trực quan hóa của mô hình hồi quy GBM.}
    \label{fig:regression_plot}
\end{figure}

\subsection{Mã nguồn cho Bài toán Phân loại}
\begin{minted}{python}
import pandas as pd
import numpy as np
import matplotlib.pyplot as plt
from sklearn.ensemble import GradientBoostingClassifier
from sklearn.preprocessing import LabelEncoder
from mlxtend.plotting import plot_decision_regions

# 1. Create DataFrame
data_cls = {
    'Likes Popcorn': ['Yes', 'Yes', 'No', 'Yes', 'No', 'No'],
    'Age': [12, 87, 44, 19, 32, 14],
    'Favorite Color': ['Blue', 'Green', 'Blue', 'Red', 'Green', 'Blue'],
    'Loves Troll 2': ['Yes', 'Yes', 'No', 'No', 'Yes', 'Yes']
}
df_cls = pd.DataFrame(data_cls)

# 2. Preprocessing (simplified for 2D visualization)
le_popcorn = LabelEncoder()
df_cls['Likes Popcorn'] = le_popcorn.fit_transform(df_cls['Likes Popcorn'])
y_encoder = LabelEncoder()
df_cls['Loves Troll 2'] = y_encoder.fit_transform(df_cls['Loves Troll 2'])

# Select 2 features for visualization
X_vis = df_cls[['Age', 'Likes Popcorn']].values
y_vis = df_cls['Loves Troll 2'].values

# 3. Build and Train the Model
gbc = GradientBoostingClassifier(n_estimators=100, learning_rate=0.1, max_depth=1, random_state=42)
gbc.fit(X_vis, y_vis)

# 4. Visualize the decision boundary
plt.style.use('seaborn-v0_8-whitegrid')
plt.figure(figsize=(10, 6))
plot_decision_regions(X_vis, y_vis, clf=gbc, legend=2)
plt.title('Gradient Boosting Classification: Decision Boundary', fontsize=16)
plt.xlabel('Age', fontsize=12)
plt.ylabel('Likes Popcorn (0: No, 1: Yes)', fontsize=12)
plt.show()
\end{minted}

\begin{figure}[H]
    \centering
    \includegraphics[width=0.9\textwidth]{projects/GradientBoosting/images/classification_plot.png}
    \caption{Đường biên quyết định của mô hình phân loại GBM.}
    \label{fig:classification_plot}
\end{figure}

\section{Tinh chỉnh siêu tham số để đạt hiệu suất tối ưu}
\begin{itemize}
    \item \textbf{\texttt{n\_estimators}}: Số lượng cây trong chuỗi.
    \item \textbf{\texttt{learning\_rate} ($\eta$)}: Tốc độ học. Có một sự đánh đổi quan trọng giữa \texttt{learning\_rate} và \texttt{n\_estimators}.
    \item \textbf{\texttt{max\_depth}}: Độ sâu tối đa của mỗi cây. Với GBM, chúng ta thường ưu tiên các cây nông (1 đến 5).
    \item \textbf{\texttt{subsample}}: Tỷ lệ mẫu dữ liệu được sử dụng để huấn luyện mỗi cây (Stochastic Gradient Boosting).
\end{itemize}

\section{So sánh với các thuật toán Ensemble khác}
\begin{table}[H]
\centering
\caption{So sánh Gradient Boosting, Random Forest, và AdaBoost.}
\begin{tabular}{l|p{3.5cm}|p{3.5cm}|p{3.5cm}}
\toprule
\textbf{Tiêu chí} & \textbf{Gradient Boosting} & \textbf{Random Forest}~\cite{breiman2001random} & \textbf{AdaBoost} \\
\midrule
\textbf{Thứ tự} & Tuần tự & Song song & Tuần tự \\
\textbf{Cơ chế học} & Học trên phần dư (gradient) & Học trên mẫu bootstrap độc lập & Học trên trọng số của các điểm bị sai \\
\textbf{Mục tiêu} & Giảm bias trước, sau đó giảm variance & Chủ yếu giảm variance & Chủ yếu giảm bias \\
\textbf{Độ sâu cây} & Cây nông & Cây sâu & Cây rất nông (stumps) \\
\bottomrule
\end{tabular}
\label{tab:comparison}
\end{table}

\section{Tổng kết}
Gradient Boosting không chỉ là một thuật toán; nó là một framework mạnh mẽ và có tính tổng quát cao~\cite{friedman2001greedy}. Bằng cách hiểu rõ cơ chế học tập theo gradient, chúng ta không chỉ nắm vững một công cụ mạnh mẽ mà còn có một nền tảng vững chắc để tiếp cận các thuật toán "hậu duệ" của nó như XGBoost~\cite{chen2016xgboost}, LightGBM và CatBoost, những "gã khổng lồ" đang thống trị nhiều cuộc thi về khoa học dữ liệu hiện nay.

% Tạo danh mục tài liệu tham khảo từ file references.bib
\bibliographystyle{plain}
\bibliography{projects/GradientBoosting/references}
\clearpage
\setcounter{section}{0} % Reset section numbering

\end{document}