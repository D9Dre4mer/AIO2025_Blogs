\begin{center}
    \LARGE{Probability in AI}
\end{center}
\begin{center}
    \large{\textit{Dao Lam Hoang}}
\end{center}

% \vspace{0.1cm}
\section{Các khái niệm cơ bản}
\subsection{Khái niệm}
\begin{itemize}
    \item \textbf{Phép thử:} Việc thực hiện một tập hợp các điều kiện cơ bản để quan sát một hiện tượng nhất định.\\
    \textit{Ví dụ:} Tung một con xúc xắc.

    \item \textbf{Không gian mẫu:} Tập hợp tất cả các kết quả có thể xảy ra.\\
    \textit{Ví dụ:} $\{1, 2, 3, 4, 5, 6\}$ là không gian mẫu khi tung xúc xắc.

    \item \textbf{Biến cố:} Là một tập con của không gian mẫu.\\
    \textit{Ví dụ:} Biến cố “ra số chẵn” là tập $\{2, 4, 6\}$.
\end{itemize}
\subsection{Biến cố}
\begin{itemize}

    \item \textbf{Biến cố:} Là một tập con của không gian mẫu, đại diện cho một hoặc nhiều kết quả mong muốn.  
    \textit{Ví dụ:} Biến cố \( A \) là "tung xúc xắc được số chẵn" thì:  
    \[
    A = \{2, 4, 6\}
    \]

    \item \textbf{Biến cố rỗng (\( \varnothing \)):} Là biến cố không chứa phần tử nào, tức là không thể xảy ra.  
    \textit{Ví dụ:} Biến cố \( B \) là "tung xúc xắc được số 8" thì:  
    \[
    B = \varnothing
    \]

    \item \textbf{Giao của hai biến cố (\( A \cap B \)):} Là biến cố xảy ra khi cả hai biến cố \( A \) và \( B \) cùng xảy ra.  
    \textit{Ví dụ:}  
    \( A = \{2, 4, 6\} \): số chẵn  
    \( B = \{4, 5, 6\} \): số lớn hơn 3  
    \[
    A \cap B = \{4, 6\}
    \]

    \item \textbf{Hợp của hai biến cố (\( A \cup B \)):} Là biến cố xảy ra khi ít nhất một trong hai biến cố \( A \) hoặc \( B \) xảy ra.  
    \textit{Ví dụ:}  
    \( A = \{1, 2, 3\} \), \( B = \{3, 4, 5\} \)  
    \[
    A \cup B = \{1, 2, 3, 4, 5\}
    \]

    \item \textbf{Hai biến cố xung khắc:} Là hai biến cố không thể xảy ra đồng thời, nghĩa là \( A \cap B = \varnothing \).  
    \textit{Ví dụ:}  
    \( A = \{1, 2\} \): số nhỏ hơn 3  
    \( B = \{4, 5, 6\} \): số lớn hơn 3  
    \[
    A \cap B = \varnothing
    \]

    \item \textbf{Biến cố đối (\( \overline{A} \)):} Là biến cố gồm tất cả các phần tử trong không gian mẫu mà không thuộc \( A \).  
    \textit{Ví dụ:}  
    Không gian mẫu \( \Omega = \{1, 2, 3, 4, 5, 6\} \)  
    \( A = \{1, 2, 3\} \)  
    \[
    \overline{A} = \{4, 5, 6\}
    \]

    \item \textbf{Biến cố độc lập:} Hai biến cố \( A \) và \( B \) được gọi là độc lập nếu:  
    \[
    P(A \cap B) = P(A) \cdot P(B)
    \]  
    \textit{Ví dụ:} Tung hai con xúc xắc.  
    - \( A \): "con thứ nhất ra số chẵn" \( \Rightarrow P(A) = \frac{3}{6} = \frac{1}{2} \)  
    - \( B \): "con thứ hai ra số 5" \( \Rightarrow P(B) = \frac{1}{6} \)  
    Vì hai xúc xắc độc lập:  
    \[
    P(A \cap B) = P(A) \cdot P(B) = \frac{1}{2} \cdot \frac{1}{6} = \frac{1}{12}
    \]

\end{itemize}





\section{Xác suất và các tính chất cơ bản}
\subsection{Khái niệm xác suất}
\begin{tcolorbox}[title= ,coltitle =black,fonttitle=\large\bfseries, colback=green!5!white,colframe=green!75!black]
Xác suất của một biến cố là khả năng xảy ra của biến cố đó.  
Ký hiệu: \( P(A) \), với \( A \) là một biến cố. \\
Các tính chất cơ bản:
\begin{itemize}
    \item \( 0 \leq P(A) \leq 1 \)
    \item \( P(\Omega) = 1 \) (biến cố chắc chắn)
    \item \( P(\varnothing) = 0 \) (biến cố không thể xảy ra)
\end{itemize}
\end{tcolorbox}


\subsection{Xác suất cổ điển}
\begin{tcolorbox}[title= ,coltitle =black,fonttitle=\large\bfseries, colback=green!5!white,colframe=green!75!black]
Nếu một thí nghiệm có \( n \) kết quả đồng khả năng xảy ra, và biến cố \( A \) có \( m \) kết quả thuận lợi, thì:
\[
P(A) = \frac{m}{n}
\]
\end{tcolorbox}

\textbf{Ví dụ:} Tung một con xúc xắc, không gian mẫu: \( \Omega = \{1, 2, 3, 4, 5, 6\} \).  
Biến cố \( A \): "ra số chẵn" \( = \{2, 4, 6\} \Rightarrow m = 3, n = 6 \)  
\[
P(A) = \frac{3}{6} = \frac{1}{2}
\]

\subsection{Xác suất có điều kiện}
\begin{tcolorbox}[title= ,coltitle =black,fonttitle=\large\bfseries, colback=green!5!white,colframe=green!75!black]
Xác suất biến cố \( A \) xảy ra với điều kiện \( B \) đã xảy ra:
\[
P(A|B) = \frac{P(A \cap B)}{P(B)} \quad \text{với } P(B) > 0
\]
\end{tcolorbox}
\textbf{Ví dụ:}  
\( A = \{2, 4, 6\} \): "ra số chẵn"  
\( B = \{4, 5, 6\} \): "ra số lớn hơn 3"  
\[
A \cap B = \{4, 6\}, \quad P(A \cap B) = \frac{2}{6}, \quad P(B) = \frac{3}{6}
\]
\[
P(A|B) = \frac{\frac{2}{6}}{\frac{3}{6}} = \frac{2}{3}
\]

\subsection{Các phép tính trong xác suất}


\subsubsection{Quy tắc cộng (Additive Rule)}
\begin{tcolorbox}[title= ,coltitle =black,fonttitle=\large\bfseries, colback=green!5!white,colframe=green!75!black]
Nếu \( A \) và \( B \) là hai biến cố bất kỳ, thì:
\[
P(A \cup B) = P(A) + P(B) - P(A \cap B)
\]

\textbf{Trường hợp đặc biệt:} Nếu \( A \cap B = \varnothing \) (hai biến cố xung khắc), thì:
\[
P(A \cup B) = P(A) + P(B)
\]
\end{tcolorbox}
\textbf{Ví dụ:} Một lớp học có 20 học sinh nam và 15 học sinh nữ. Chọn ngẫu nhiên một học sinh.

Gọi \( A \): "chọn được học sinh nam"  
Gọi \( B \): "chọn được học sinh nữ"  

Vì mỗi học sinh thuộc chỉ một giới tính, nên \( A \cap B = \varnothing \), và:
\[
P(A) = \frac{20}{35}, \quad P(B) = \frac{15}{35}
\]
\[
P(A \cup B) = P(A) + P(B) = \frac{20 + 15}{35} = 1
\]


\subsubsection{Quy tắc nhân (Multiplicative Rule)}
\begin{tcolorbox}[title= ,coltitle =black,fonttitle=\large\bfseries, colback=green!5!white,colframe=green!75!black]
\begin{itemize}
    \item \textbf{Quy tắc tích tổng quát:}
    \[
    P(A_1, A_2, \ldots, A_n) = P(A_1) \cdot P(A_2|A_1) \cdot P(A_3|A_1, A_2) \cdots P(A_n|A_1, A_2, \ldots, A_{n-1})
    \]

    \item \textbf{Công thức xác suất giao khi biết xác suất có điều kiện:}
    \[
    P(A \cap B) = P(A|B) \cdot P(B)
    \]

    \item \textbf{Nếu \( A \) và \( B \) độc lập, thì:}
    \[
    P(A \cap B) = P(A) \cdot P(B)
    \]
\end{itemize}
\end{tcolorbox}

\textbf{Ví dụ:} Một hộp có 3 bi đỏ và 2 bi xanh. Lấy lần lượt 2 viên bi không hoàn lại.

Gọi:
- \( A \): "bi đầu là đỏ"
- \( B \): "bi thứ hai là đỏ"

Ta có:
\[
P(A) = \frac{3}{5}, \quad P(B|A) = \frac{2}{4}
\Rightarrow P(A \cap B) = P(A) \cdot P(B|A) = \frac{3}{5} \cdot \frac{2}{4} = \frac{3}{10}
\]


\section{Định lý xác suất toàn phần và Định lý Bayes}

\subsection{Định lý xác suất toàn phần}
\begin{tcolorbox}[title= ,coltitle =black,fonttitle=\large\bfseries, colback=green!5!white,colframe=green!75!black]
Giả sử không gian mẫu \( \Omega \) được chia thành các biến cố \( B_1, B_2, \ldots, B_n \) sao cho:
\begin{itemize}
  \item \( B_i \cap B_j = \varnothing \quad \text{với } i \ne j \) (các biến cố không giao nhau),
  \item \( \bigcup_{i=1}^n B_i = \Omega \) (phủ toàn bộ không gian mẫu),
  \item \( P(B_i) > 0 \quad \text{với mọi } i \).
\end{itemize}

Khi đó, với mọi biến cố \( A \subset \Omega \), ta có công thức:
\[
P(A) = \sum_{i=1}^{n} P(B_i) \cdot P(A \mid B_i)
\]
\end{tcolorbox}
\textbf{Ví dụ:} Có ba hộp bi:
\begin{itemize}
    \item Hộp 1: 2 bi đỏ, 3 bi xanh
    \item Hộp 2: 4 bi đỏ, 1 bi xanh
    \item Hộp 3: 3 bi đỏ, 3 bi xanh
\end{itemize}
Chọn ngẫu nhiên một hộp (xác suất chọn mỗi hộp là \( \frac{1}{3} \)), sau đó lấy ngẫu nhiên 1 viên bi từ hộp đó.

Tính xác suất lấy được bi đỏ.

Gọi:
\begin{itemize}
    \item \( B_1, B_2, B_3 \): chọn hộp 1, 2, 3 tương ứng
    \item \( A \): biến cố "lấy được bi đỏ"
\end{itemize}

Ta có:
\[
P(A) = \sum_{i=1}^3 P(B_i) \cdot P(A \mid B_i)
= \frac{1}{3} \cdot \frac{2}{5}
+ \frac{1}{3} \cdot \frac{4}{5}
+ \frac{1}{3} \cdot \frac{3}{6}
= \frac{1}{3} \left( \frac{2}{5} + \frac{4}{5} + \frac{1}{2} \right)
= \frac{1}{3} \cdot \left( \frac{6}{5} + \frac{1}{2} \right)
= \frac{1}{3} \cdot \frac{17}{10}
= \frac{17}{30}
\]

\subsection{Định lý Bayes}
\begin{tcolorbox}[title= ,coltitle =black,fonttitle=\large\bfseries, colback=green!5!white,colframe=green!75!black]
Nếu \( B_1, B_2, \ldots, B_n \) không giao nhau, và phủ toàn bộ \( \Omega \), với \( P(B_i) > 0 \), và \( A \) là biến cố có \( P(A) > 0 \), thì xác suất có điều kiện của \( B_k \) khi biết \( A \) được tính bởi:

\[
P(B_k \mid A) = \frac{P(B_k) \cdot P(A \mid B_k)}{\sum_{i=1}^{n} P(B_i) \cdot P(A \mid B_i)}
\]
\end{tcolorbox}

\textbf{Ví dụ:} Tiếp tục từ ví dụ trên, giả sử ta đã rút được một viên bi đỏ. Tính xác suất viên bi đó đến từ hộp 2.

Ta cần tính \( P(B_2 \mid A) \), với:
\[
P(B_2 \mid A) = \frac{P(B_2) \cdot P(A \mid B_2)}{P(A)}
= \frac{\frac{1}{3} \cdot \frac{4}{5}}{\frac{17}{30}}
= \frac{\frac{4}{15}}{\frac{17}{30}}
= \frac{4}{15} \cdot \frac{30}{17}
= \frac{120}{255}
= \frac{8}{17}
\]



\section{Simple Classification}

\subsection{Định Lý Bayes với một feature}
\begin{tcolorbox}[title= ,coltitle =black,fonttitle=\large\bfseries, colback=green!5!white,colframe=green!75!black]
Giả sử ta muốn tính xác suất của một lớp \( C = c \) dựa trên một đặc trưng đầu vào \( X = x \). Khi đó, theo định lý Bayes:

\[
P(C = c \mid X = x) = \frac{P(X = x \mid C = c) \cdot P(C = c)}{P(X = x)}
\]


Ví dụ với hai lớp:
\[
P(C = c_1 \mid X = x) = ?, \quad P(C = c_2 \mid X = x) = ?
\]
\end{tcolorbox}

\subsection{Ví dụ: Kết quả học tập}

Giả sử ta có dữ liệu xác suất về việc học sinh có học hay không và kết quả thi:

\begin{align*}
P(res = pass \mid stud = yes) &= \frac{P(stud = yes \mid res = pass) \cdot P(res = pass)}{P(stud = yes)} \\
P(res = fail \mid stud = yes) &= \frac{P(stud = yes \mid res = fail) \cdot P(res = fail)}{P(stud = yes)}
\end{align*}

\subsection{Mã Python minh họa}

\begin{minted}[fontsize=\small, breaklines]{bash}
# Giả sử dữ liệu xác suất đã biết:
P_stud_given_pass = 0.8
P_pass = 0.6
P_stud_given_fail = 0.3
P_fail = 0.4
P_stud = P_stud_given_pass * P_pass + P_stud_given_fail * P_fail

# Áp dụng định lý Bayes:
P_pass_given_stud = (P_stud_given_pass * P_pass) / P_stud
P_fail_given_stud = (P_stud_given_fail * P_fail) / P_stud

print("P(pass | stud = yes):", round(P_pass_given_stud, 3))
print("P(fail | stud = yes):", round(P_fail_given_stud, 3))
\end{minted}

\textbf{Kết quả:}  
Nếu sinh viên học (\texttt{stud = yes}), thì:
\[
P(pass \mid stud = yes) \approx 0.8, \quad P(fail \mid stud = yes) \approx 0.2
\]

