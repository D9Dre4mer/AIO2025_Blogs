\documentclass[12pt]{article}
% Các gói cơ bản cho tiếng Việt và font chữ
\usepackage{fontspec}
\usepackage{polyglossia}
\setmainlanguage{vietnamese}
\setotherlanguage{english}
\setmainfont{Times New Roman} % Hoặc font bạn ưa thích, ví dụ: Arial, Cambria

% Gói cho tiêu đề phụ
\usepackage{titling}

% Các gói toán học, đồ họa, bảng biểu
\usepackage{amsmath, amssymb, graphicx, float}
\usepackage{xcolor}
\usepackage{tabularx}
\usepackage{booktabs} % Cho bảng biểu đẹp hơn
\usepackage{caption} % Tùy chỉnh caption
\captionsetup[figure]{labelsep=period, font=small, labelfont=bf}
\captionsetup[table]{labelsep=period, font=small, labelfont=bf}


% Các gói định dạng văn bản
\usepackage{enumitem}
\usepackage[hidelinks]{hyperref}
\usepackage{setspace}
\usepackage{microtype}

% Thiết lập giãn dòng 1.5
\onehalfspacing
\tolerance=1000
\emergencystretch=3em

% Thiết lập lề trang
\usepackage[a4paper, total={6in, 8in}]{geometry}

% Thiết lập cho tiêu đề các section
\usepackage{titlesec}
\titleformat{\section}
  {\normalfont\Large\bfseries}{\thesection}{1em}{}
\titleformat{\subsection}
  {\normalfont\large\bfseries}{\thesubsection}{1em}{}

% Thiết lập cho các khối mã nguồn (sử dụng gói listings)
\usepackage{listings}

\definecolor{backcolour}{rgb}{0.95,0.95,0.92}
\definecolor{codegreen}{rgb}{0,0.6,0}
\definecolor{codegray}{rgb}{0.5,0.5,0.5}
\definecolor{codepurple}{rgb}{0.58,0,0.82}
\definecolor{keywordblue}{rgb}{0.0, 0.0, 0.9}

\lstset{
    language=Python,
    basicstyle=\ttfamily\small,
    keywordstyle=\color{keywordblue},
    commentstyle=\color{codegreen},
    stringstyle=\color{codepurple},
    backgroundcolor=\color{backcolour},
    numbers=left,
    numberstyle=\tiny\color{codegray},
    frame=tb, % top and bottom frame
    breaklines=true,
    inputencoding=utf8,
    columns=fullflexible,
    upquote=true,
    captionpos=b,
    showspaces=false,
    showstringspaces=false,
    tabsize=2
}

\begin{document}
\begin{center}
    \LARGE{List}
\end{center}
\begin{center}
    \large{\textit{Dao Lam Hoang}}
\end{center}

% \vspace{0.1cm}
\section{List là gì}
- \textbf{List} là một tập hợp chứa các phần tử, có thể trùng nhau và kiểu dữ liệu khác nhau.
\begin{lstlisting}[language=python]
# danh sách trống
empty_list = []

# danh sách số tự nhiên nhỏ hơn 10
my_list = list(range(10))

# danh sách kết hợp nhiều kiểu dữ liệu
mixed_list = [True, 5, 'some string', 123.45]
n_list = ["Happy", [2, 0, 1, 5]]

# danh sách các loại hoa quả (đồng nhất tiếng Việt)
shopping_list = ['táo', 'chuối', 'anh đào', 'dâu', 'mận']
    
\end{lstlisting}

\section{Các lệnh của List}
\subsection{Slicing}
- list[start:end:step]
\begin{lstlisting}[language = python]
data = [4, 5, 6, 7, 8, 9]
print(data[:3])  # Lấy từ đầu đến phần tử thứ 3 
print(data[2:4]) # Lấy từ phần tử thứ 2 đến trước phần tử thứ 4
print(data[3:])  # Lấy từ phần tử thứ 3 đến hết
\end{lstlisting}
\subsection{Các syntax}
\begin{lstlisting}[language = python]
data = [6, 5, 7, 1, 9, 2]

# Thêm phần tử
data.append(4)             # Thêm 4 vào cuối: [6, 5, 7, 1, 9, 2, 4]
data.insert(0, 4)          # Thêm 4 vào vị trí index = 0
data.extend([9, 2])        # Thêm nhiều phần tử vào cuối: [9, 2]

# Thay đổi phần tử
data[1] = 4                # Thay phần tử ở index = 1 bằng 4

# Xóa phần tử
data.pop(2)               # Xóa phần tử tại index = 2
data.remove(5)            # Xóa phần tử đầu tiên có giá trị là 5
del data[1:3]             # Xóa phần tử tại index 1 và 2
data.clear()              # Xóa toàn bộ danh sách

# Thao tác khác
data = [6, 5, 7, 1, 9, 2] # Reset lại dữ liệu để tiếp tục thao tác
data.reverse()            # Đảo ngược danh sách
count_7 = data.count(7)   # Đếm số lần xuất hiện của giá trị 7

data_copy = data.copy()   # Sao chép danh sách
data.sort()               # Sắp xếp tăng dần
data.sort(reverse=True)   # Sắp xếp giảm dần

# Nối và nhân danh sách
data1 = [1, 2]
data2 = [3, 4]
data = data1 + data2      # Nối 2 danh sách: [1, 2, 3, 4]

data_m = data * 3        
# Nhân danh sách với số nguyên: [1, 2, 3, 4, 1, 2, 3, 4, 1, 2, 3, 4]

\end{lstlisting}

\subsection{Các hàm có sẵn cho List}
\begin{lstlisting}[language = python]
data = [6, 5, 7, 1, 9, 2]

print(len(data))         # Độ dài danh sách → 6
print(min(data))         # Giá trị nhỏ nhất → 1
print(max(data))         # Giá trị lớn nhất → 9
print(sum(data))         # Tổng các phần tử → 30

# zip với ví dụ cụ thể
a = [1, 2, 3]
b = ['a', 'b', 'c']
print(list(zip(a, b)))  
# Ghép cặp phần tử tương ứng → [(1, 'a'), (2, 'b'), (3, 'c')]

# reversed
print(list(reversed(data)))  
# Đảo ngược danh sách → [2, 9, 1, 7, 5, 6]

# sorted
print(sorted(data))      
# Sắp xếp tăng dần → [1, 2, 5, 6, 7, 9]

# enumerate
for i, value in enumerate(data):
    print(i, value)      # In chỉ số và giá trị từng phần tử


\end{lstlisting}













\end{document}

