 \documentclass[a4paper,12pt]{article}

% Encoding và font cho tiếng Việt
\usepackage{polyglossia}
\setmainlanguage{vietnamese}
\setotherlanguage{english}

\usepackage{fontspec}
\setmainfont{Times New Roman}
\setsansfont{Times New Roman}
\setmonofont{Times New Roman}[Scale=MatchLowercase]

% Toán học và ký hiệu
\usepackage{amsmath, amssymb, amsfonts}
\usepackage{pifont} % For \ding commands
\usepackage{bbding} % Alternative dingbat symbols

% Cấu hình trang
\usepackage{geometry}
\geometry{left=2.5cm, right=2.5cm, top=2.5cm, bottom=2.5cm}

% Header/Footer
\usepackage{fancyhdr}
\setlength{\headheight}{40 pt}
\pagestyle{fancy}
\fancyhead[LO,L]{GRID034 (AIO)}
\fancyhead[CO,C]{}
\fancyhead[RO,R]{\today}
\fancyfoot[CO,C]{\thepage}

% Hình ảnh
\usepackage{graphicx}
\usepackage{subcaption}

% Minted - mã lệnh có hỗ trợ tiếng Việt tốt hơn
\usepackage{minted}
\usepackage{float}

% Màu sắc cho minted
\usepackage{xcolor}
\definecolor{codebg}{rgb}{0.97,0.97,0.97}

% Cấu hình minted
\setminted{
    fontsize=\footnotesize,
    breaklines=true,
    breakanywhere=true,
    linenos=true,
    frame=lines,
    framesep=2mm,
    bgcolor=codebg
}

% Hyperref setup
\usepackage[hidelinks]{hyperref}
\hypersetup{
    colorlinks=true,
    linkcolor=blue,
    filecolor=magenta,
    urlcolor=cyan
}

% Các package bổ sung
\usepackage{parskip}
\usepackage{enumitem}
\usepackage{longtable}
\usepackage{booktabs}
\usepackage{makecell}
\usepackage{algorithm}
\usepackage{algpseudocode}
\usepackage{cancel}
\usepackage{titlesec}
\usepackage{blindtext}
\usepackage{datetime}
\usepackage{placeins}
\usepackage{array}
\usepackage{tabularx}
\usepackage{multirow}
\usepackage[table]{xcolor}
\usepackage[bottom]{footmisc}
\usepackage{tcolorbox}  % For the tcolorbox environment
\usepackage{tikz}
\usetikzlibrary{positioning, arrows.meta, shapes.geometric}

% Unicode special characters
\usepackage{newunicodechar}

% Custom command
\newcommand{\code}[1]{\texttt{#1}}

% Date format
\newdateformat{mydate}{\THEDAY/\THEMONTH/\THEYEAR}

% Section formatting
\titleformat{\section}{\large\bfseries}{\thesection.}{0.5em}{}

% Unicode characters
\newunicodechar{✔}{\checkmark}
\newunicodechar{❌}{\texttimes} % approximate
\newunicodechar{⇒}{\ensuremath{\Rightarrow}}

% Thông tin tài liệu
\author{}
\date{\mydate\today} % Gọi preamble bạn đã định nghĩa

\title{\textbf{Dự báo Doanh số với Explainable AI (XAI):\\
Một Case Study kết hợp LightGBM và SHAP}}
\author{Vương Nguyệt Bình}
\date{\mydate\today} % Dùng định dạng ngày trong preamble

\begin{document}
\maketitle

\begin{abstract}
Dự báo doanh số bán hàng là một thách thức lớn đối với doanh nghiệp,
khi mà hành vi tiêu dùng chịu ảnh hưởng đồng thời từ nhiều yếu tố như mùa vụ,
ngày lễ, thời tiết, và xu hướng thị trường.  

Trong nghiên cứu này, chúng tôi xây dựng hệ thống dự báo dựa trên thuật toán
\textbf{LightGBM}, kết hợp với \textbf{SHAP} để giải thích kết quả dự báo,
sử dụng \textbf{Optuna} cho tối ưu tham số và triển khai qua ứng dụng web \textbf{Streamlit}.  

Điểm nổi bật là hệ thống vừa đạt độ chính xác cao, vừa minh bạch,
tạo điều kiện cho nhà quản lý kinh doanh ra quyết định nhanh chóng và đáng tin cậy.
\end{abstract}

\section{Giới thiệu}
\subsection{Bối cảnh kinh doanh}
Trong ngành bán lẻ, việc dự báo doanh số đóng vai trò trung tâm. 
Một dự báo chính xác giúp doanh nghiệp:
\begin{itemize}
    \item Lập kế hoạch tồn kho, tránh thiếu hoặc thừa hàng.
    \item Bố trí nhân sự hợp lý trong các ngày cao điểm.
    \item Lên chiến lược khuyến mãi và marketing phù hợp.
\end{itemize}

\textbf{Thách thức chính}:  
Doanh số bị ảnh hưởng bởi nhiều yếu tố \textit{không tuyến tính} và \textit{khó dự đoán}, 
ví dụ: thời tiết xấu làm khách hàng ít đi mua sắm, 
ngày lễ kéo doanh số tăng vọt, hoặc xu hướng đột ngột từ mạng xã hội.  

\subsection{Vấn đề của mô hình “hộp đen”}
Nhiều thuật toán AI hiện đại cho kết quả chính xác nhưng không thể giải thích được.  
Nhà quản lý kinh doanh cần biết không chỉ “dự báo là bao nhiêu”, 
mà còn “tại sao lại như vậy”.  

\begin{quote}
\textit{Ví dụ:} Nếu hệ thống báo rằng tuần tới doanh số sẽ giảm 15\%, 
một nhà quản lý sẽ đặt câu hỏi:  
“Giảm là do thời tiết, do đối thủ cạnh tranh, hay do mùa vụ?”
\end{quote}

\subsection{Mục tiêu dự án}
Dự án này được thiết kế với hai nhóm mục tiêu chính: 
\textbf{mục tiêu kinh doanh} và \textbf{mục tiêu kỹ thuật}. 
Điều này đảm bảo rằng nỗ lực kỹ thuật không tách rời giá trị mà doanh nghiệp nhận được. 

\paragraph{Mục tiêu Kinh doanh:}
\begin{itemize}
  \item \textbf{Độ chính xác}: đạt được mức dự báo cao, đủ để cải thiện kế hoạch và giảm lãng phí.
  \item \textbf{Khả năng giải thích}: cung cấp lý do đằng sau mỗi con số, giúp nhà quản lý tự tin hơn khi ra quyết định.
  \item \textbf{Khả năng mở rộng}: dễ dàng áp dụng cho nhiều chi nhánh hoặc cửa hàng khác nhau.
  \item \textbf{Tính khả dụng}: giao diện thân thiện, để cả nhân viên không chuyên về kỹ thuật cũng sử dụng được.
\end{itemize}

\paragraph{Mục tiêu Kỹ thuật:}
\begin{itemize}
  \item \textbf{Hiệu suất mô hình}: giảm sai số RMSE, MAE, MAPE, vượt qua baseline Prophet.
  \item \textbf{Kỹ thuật đặc trưng}: tạo ra bộ đặc trưng phong phú (thời gian, lag, rolling statistics...).
  \item \textbf{Khả năng giải thích}: cung cấp cả phân tích SHAP toàn cục (global) và cục bộ (local).
  \item \textbf{Sẵn sàng sản xuất}: mã nguồn module hóa, dễ bảo trì, có tài liệu đầy đủ.
\end{itemize}

Điểm thú vị là các mục tiêu này không tách biệt, 
mà bổ sung cho nhau: 
một mô hình chính xác nhưng khó hiểu thì không có giá trị, 
một mô hình dễ hiểu nhưng kém chính xác cũng vô dụng. 
Dự án này chọn con đường dung hòa cả hai.

\textbf{Gợi ý hình minh hoạ:} Bảng tóm tắt hai nhóm mục tiêu: 
cột trái là kinh doanh (accuracy, interpretability, scalability, usability), 
cột phải là kỹ thuật (RMSE, feature engineering, SHAP, production readiness).

\section{Các công nghệ cốt lõi sử dụng}

Không phải ngẫu nhiên mà dự án chọn LightGBM, SHAP, Optuna và Streamlit. 
Mỗi công nghệ đều được lựa chọn để giải quyết một “nút thắt” cụ thể. 
Cùng điểm qua:

\paragraph{LightGBM – Trái tim của mô hình dự báo}
LightGBM là một framework \emph{gradient boosting} nổi tiếng, 
được tối ưu cho tốc độ và hiệu suất, đặc biệt phù hợp với dữ liệu dạng bảng (tabular data).  
Với hơn 50 đặc trưng được tạo ra, LightGBM xử lý nhanh chóng, 
khai thác được cả mối quan hệ phi tuyến tính trong dữ liệu.  
So với Prophet, LightGBM cho khả năng học sâu hơn từ các đặc trưng phức tạp, 
tạo ra dự báo chính xác hơn.

\paragraph{SHAP – Khi AI biết giải thích}
SHAP (\emph{SHapley Additive exPlanations}) là công cụ biến mô hình “hộp đen” 
thành một mô hình \textbf{có thể giải thích được}.  
Nó dựa trên lý thuyết trò chơi để tính giá trị Shapley cho từng đặc trưng, 
nói cách khác: “mỗi yếu tố đóng góp bao nhiêu phần vào kết quả cuối cùng?”.  
Nhờ SHAP, các nhà quản lý không chỉ thấy con số dự báo, 
mà còn hiểu vì sao mô hình đưa ra kết quả đó.

\paragraph{Optuna – Tối ưu hoá thông minh}
Huấn luyện LightGBM không chỉ là cho dữ liệu vào rồi bấm nút.  
Hiệu suất của nó phụ thuộc mạnh mẽ vào các siêu tham số (hyperparameters) như 
\texttt{num\_leaves}, \texttt{learning\_rate}, \texttt{reg\_alpha}, \texttt{reg\_lambda}...  
Thay vì thử tay thủ công, nhóm dùng \textbf{Optuna} – một framework tối ưu hoá tự động.  
Nó giúp thử nghiệm hàng trăm cấu hình khác nhau và chọn ra bộ tham số tốt nhất, 
tiết kiệm rất nhiều thời gian và công sức.

\paragraph{Streamlit – Cầu nối đến người dùng cuối}
Một mô hình AI mạnh đến đâu mà không ai dùng thì cũng vô nghĩa.  
Streamlit giúp biến mô hình thành một \textbf{ứng dụng web tương tác}, 
nơi người dùng (kể cả không rành kỹ thuật) có thể:
\begin{itemize}
  \item Xem dữ liệu lịch sử bán hàng.
  \item Nhận dự báo doanh số cho tương lai gần.
  \item Xem giải thích trực quan (biểu đồ SHAP) cho từng dự đoán.
\end{itemize}
Điều này biến AI từ một “món đồ trong phòng lab” thành một \textbf{công cụ kinh doanh thực thụ}.

\paragraph{Các thư viện hỗ trợ khác}
Đằng sau các công nghệ chính, còn có cả một hệ sinh thái thư viện hỗ trợ:
\begin{itemize}
  \item \textbf{Pandas, NumPy}: xử lý dữ liệu nhanh và hiệu quả.
  \item \textbf{Scikit-learn}: cung cấp công cụ chuẩn hóa và hỗ trợ huấn luyện.
  \item \textbf{Matplotlib, Plotly}: trực quan hóa dữ liệu và kết quả.
\end{itemize}

\bigskip
Sự kết hợp này thể hiện một tư duy trưởng thành trong việc xây dựng sản phẩm AI: 
không chỉ tập trung vào mô hình, mà nhìn vào \textbf{toàn bộ vòng đời giải pháp}, 
từ dữ liệu đến người dùng cuối.

% --- Các section tiếp theo ở đây ---
% (Quy trình triển khai, Kết quả, Bài học, Hướng phát triển, Kết luận...)
\section{Quy trình triển khai}
\subsection{Tổng quan pipeline}
\begin{figure}[H] % [H], [h], [t], [b] để điều khiển vị trí
    \centering
    \includegraphics[width=0.8\textwidth]{images/pipeline.png}
    \caption{Các bước triển khai dự án}
    \label{fig:ten_nhan}
\end{figure}
Nếu coi AI như một “đầu bếp dữ liệu”, thì quy trình triển khai mô hình dự báo 
chính là công thức nấu ăn. 
Một món ăn ngon không chỉ phụ thuộc vào nguyên liệu, 
mà còn ở cách chuẩn bị, cách nấu, và cách trình bày.  
Dự án này cũng vậy: từ dữ liệu thô đến mô hình hoạt động, 
mọi bước đều được thiết kế cẩn thận.  

\subsection{Bước 1: Tiền xử lý Dữ liệu}

Mọi mô hình tốt đều bắt đầu từ dữ liệu tốt.  
Nhóm tích hợp hai nguồn dữ liệu chính:
\begin{itemize}
  \item \textbf{Dữ liệu bán hàng} (2016–2017): bao gồm doanh số và số lượng giao dịch.
  \item \textbf{Dữ liệu thời tiết}: các yếu tố như nhiệt độ, độ ẩm, lượng mưa.
\end{itemize}

Sau khi gom dữ liệu, công việc quan trọng là \textbf{làm sạch}:  
\begin{itemize}
  \item \emph{Giá trị thiếu}: điền bằng trung bình, trung vị hoặc nội suy.
  \item \emph{Ngoại lệ}: phát hiện bằng IQR (interquartile range) hoặc Z-score và xử lý.
\end{itemize}

Mục tiêu của bước này: biến dữ liệu thô thành một bộ dữ liệu \textbf{sạch, nhất quán, đáng tin cậy}.

\subsection{Bước 2: Phân tích Khám phá Dữ liệu (EDA)}

EDA giống như giai đoạn “làm quen với nguyên liệu”.  
Nhóm phân tích dữ liệu để tìm ra:
\begin{itemize}
  \item \textbf{Tính mùa vụ}: ví dụ, doanh số cao vào một số tháng nhất định.
  \item \textbf{Xu hướng dài hạn}: liệu doanh số có tăng đều qua các năm?
  \item \textbf{Tương quan}: giữa thời tiết (nhiệt độ, mưa) và doanh số bán hàng.
\end{itemize}

Kết quả của EDA giúp xác định các tín hiệu ẩn trong dữ liệu, 
đồng thời cung cấp trực giác quan trọng để thiết kế đặc trưng và chọn mô hình.


\subsection{Bước 3: Kỹ thuật Tạo Đặc trưng (Feature Engineering)}

Nếu dữ liệu là nguyên liệu, thì đặc trưng chính là “gia vị” quyết định độ ngon.  
Dự án tạo ra hơn \textbf{50 đặc trưng}, bao gồm:

\begin{itemize}
  \item \textbf{Đặc trưng thời gian}: ngày trong tuần, tháng, quý, ngày lễ, cuối tuần.
  \item \textbf{Lag features (độ trễ)}: doanh số của 1, 7, 14, 28 ngày trước.
  \item \textbf{Thống kê cuộn (rolling statistics)}: trung bình động 7 ngày, độ lệch chuẩn 14 ngày.
\end{itemize}

Nhờ đó, mô hình không chỉ nhìn thấy “ngày hôm nay”, 
mà còn hiểu được ngữ cảnh lịch sử và quy luật ẩn sau dữ liệu.  


\subsection{Bước 4: Phát triển Mô hình}

Nhóm áp dụng chiến lược hai tầng:
\begin{itemize}
  \item \textbf{Prophet}: dùng làm baseline. Đây là mô hình phổ biến cho chuỗi thời gian, xử lý tốt mùa vụ, dễ triển khai.
  \item \textbf{LightGBM}: mô hình nâng cao, tận dụng đặc trưng phức tạp và phi tuyến tính để vượt qua Prophet.
\end{itemize}

Để đảm bảo kết quả khách quan, dự án dùng \textbf{time series splitting}, 
tránh việc rò rỉ dữ liệu từ tương lai về quá khứ (data leakage).  

Kết quả: LightGBM vượt trội hơn Prophet, 
chứng minh giá trị của việc đầu tư vào đặc trưng và mô hình phi tuyến.

\subsection{Bước 5: Tối ưu hoá Siêu tham số (Hyperparameter Tuning)}

Mô hình mạnh chưa đủ, cần được “điều chỉnh” tinh vi.  
Optuna được sử dụng để tự động hóa quá trình này:  
\begin{itemize}
  \item Thử hơn 100 cấu hình tham số khác nhau.
  \item Tối ưu các tham số quan trọng như \texttt{num\_leaves}, \texttt{learning\_rate}, \texttt{reg\_alpha}, \texttt{reg\_lambda}.
  \item Áp dụng cross-validation cho chuỗi thời gian để tránh overfitting.
\end{itemize}

Kết quả là một mô hình \textbf{nhẹ hơn, nhanh hơn, chính xác hơn}, 
sẵn sàng để bước sang giai đoạn triển khai thực tế.

\section{Phân tích Khả năng Giải thích (XAI)}

Một mô hình AI giỏi có thể nói cho bạn biết \emph{“ngày mai sẽ bán được 1.200 sản phẩm”}.  
Nhưng một mô hình AI giỏi \textbf{và minh bạch} sẽ còn nói thêm:  
\emph{“1.200 sản phẩm vì thời tiết đẹp, tuần trước bán chạy, và hôm nay rơi vào dịp lễ”}.  

Đây chính là giá trị cốt lõi của \textbf{XAI – Explainable AI}, 
giúp biến dự đoán thành một \textbf{câu chuyện dễ hiểu và hành động được}.  

Trong dự án này, nhóm đã dùng \textbf{SHAP} để phân tích mô hình LightGBM, 
với hai góc nhìn: toàn cục và cục bộ.  

\subsection{Giải thích Toàn cục (Global Explanations)}

SHAP toàn cục cho một bức tranh tổng quan:  
những yếu tố nào ảnh hưởng mạnh nhất đến doanh số?  

Ví dụ:
\begin{itemize}
  \item Ngày lễ và cuối tuần thường đẩy doanh số tăng cao.
  \item Thời tiết xấu (mưa nhiều) làm doanh số giảm đáng kể.
  \item Doanh số tuần trước là tín hiệu mạnh để dự báo tuần tới.
\end{itemize}

Với thông tin này, nhà quản lý có thể xây dựng chiến lược dài hạn:  
nếu dữ liệu cho thấy thời tiết có ảnh hưởng mạnh, 
thì các cửa hàng nên tăng cường khuyến mãi online vào mùa mưa, 
thay vì cố gắng kéo khách đến cửa hàng vật lý.


\subsection{Giải thích Cục bộ (Local Explanations)}

Nếu toàn cục là cái nhìn từ trên cao, 
thì cục bộ chính là soi kính lúp vào từng dự đoán riêng lẻ.  

Ví dụ: mô hình dự báo rằng \emph{“Ngày mai, cửa hàng A sẽ bán được ít hơn bình thường”}.  
Nhưng tại sao?  
Nhờ SHAP, nhà quản lý thấy rõ:
\begin{itemize}
  \item “Doanh số tuần trước giảm” đóng góp -150 sản phẩm.
  \item “Thời tiết xấu” đóng góp -80 sản phẩm.
  \item “Hôm nay là thứ 7” đóng góp +30 sản phẩm.
\end{itemize}

Tổng hợp lại, kết quả dự báo thấp hơn bình thường là hợp lý.  
Và quan trọng hơn: người quản lý có thể \textbf{hành động}, 
ví dụ tung thêm khuyến mãi cuối tuần để bù lại tác động tiêu cực từ thời tiết.  


\subsection{Từ con số sang hành động}

Khả năng giải thích biến mô hình dự báo từ một “máy tính toán” 
thành một \textbf{công cụ ra quyết định}.  
Người dùng không chỉ biết \emph{“chuyện gì sẽ xảy ra”}, 
mà còn hiểu \emph{“tại sao”} và \emph{“cần làm gì”}.  

Chẳng hạn:
\begin{itemize}
  \item Nếu doanh số dự báo giảm do \textbf{thời tiết}, giải pháp có thể là tăng quảng bá online.
  \item Nếu do \textbf{thiếu ngày lễ}, có thể bù bằng các chương trình khuyến mãi nhân tạo “mini-event”.
  \item Nếu do \textbf{xu hướng giảm dài hạn}, cần xem xét lại chiến lược sản phẩm.
\end{itemize}

Nói cách khác, XAI không chỉ làm cho AI đáng tin cậy hơn, 
mà còn khiến nó trở thành một \textbf{người đồng hành chiến lược} trong doanh nghiệp.


\section{Kết quả và Thảo luận}

\subsection{Hiệu suất Mô hình}

Khi so sánh với Prophet – mô hình baseline phổ biến cho chuỗi thời gian – 
LightGBM cho thấy sự vượt trội rõ rệt.  
Trên mọi chỉ số đánh giá (RMSE, MAE, MAPE), 
LightGBM đều cho sai số thấp hơn.  

Điều này chứng minh rằng việc đầu tư vào kỹ thuật tạo đặc trưng, 
cùng với khả năng xử lý phi tuyến của LightGBM, 
mang lại giá trị thiết thực vượt xa các mô hình truyền thống.  

\textbf{Gợi ý hình minh hoạ:} Biểu đồ cột so sánh RMSE, MAE, MAPE giữa Prophet và LightGBM.

\subsection{Phân tích SHAP}

Bên cạnh độ chính xác, SHAP cho ta cái nhìn sâu sắc hơn về cách mô hình ra quyết định:

\begin{itemize}
  \item \textbf{Mùa vụ} chiếm hơn 40\% ảnh hưởng đến dự báo – minh chứng rằng các dịp lễ, cuối tuần đóng vai trò quyết định.  
  \item \textbf{Lag features} (dữ liệu quá khứ) giúp mô hình nắm bắt được chu kỳ tiêu dùng, đặc biệt là xu hướng “tuần trước ảnh hưởng tuần sau”.  
  \item \textbf{Thời tiết} không phải yếu tố chính, nhưng vẫn góp phần đáng kể trong nhiều trường hợp (ví dụ, mưa làm giảm doanh số trực tiếp).  
\end{itemize}

Những phát hiện này vừa xác nhận trực giác kinh doanh (ngày lễ quan trọng), 
vừa hé lộ thêm chi tiết ẩn (chu kỳ tiêu dùng từ lag features).  
\begin{figure}[H] % [H], [h], [t], [b] để điều khiển vị trí
    \centering
    \includegraphics[width=0.8\textwidth]{ten_file_anh.png}
    \caption{Chú thích ảnh}
    \label{fig:ten_nhan}
\end{figure}

\subsection{Ứng dụng Thực tế}

Từ góc nhìn kinh doanh, mô hình không chỉ đưa ra dự báo, 
mà còn biến thành công cụ hỗ trợ ra quyết định.  

Người quản lý có thể:
\begin{itemize}
  \item \textbf{Xem dự báo kèm giải thích minh bạch}: không còn là “con số từ hộp đen”.  
  \item \textbf{Lập kế hoạch tồn kho chính xác hơn}: giảm lãng phí và thiếu hụt.  
  \item \textbf{Ra quyết định dựa trên dữ liệu thay vì cảm tính}: tăng độ tin cậy và tính bền vững trong quản lý.  
\end{itemize}

Nói cách khác, dự án đã vượt qua ranh giới của một mô hình dự báo thông thường, 
trở thành một công cụ \textbf{ra quyết định chiến lược}.  


\section{Hạn chế và Bài học Rút ra}

Dù dự án đạt được nhiều thành tựu, vẫn còn tồn tại những hạn chế và bài học quý giá.  
Nhìn thẳng vào những điểm này giúp doanh nghiệp có cái nhìn thực tế, 
và là cơ sở để cải tiến trong tương lai.  

\subsection{Hạn chế}

\begin{itemize}
  \item \textbf{Nguồn dữ liệu còn hạn chế}: mới chỉ sử dụng dữ liệu bán hàng và thời tiết.  
  Các yếu tố khác như dữ liệu đối thủ, chỉ số kinh tế vĩ mô, hay dữ liệu từ mạng xã hội 
  chưa được tích hợp.  
  \item \textbf{Mô hình dừng ở LightGBM}: mặc dù mạnh mẽ, nhưng dự án chưa thử nghiệm các kiến trúc 
  deep learning (LSTM, Transformer) vốn có lợi thế với chuỗi thời gian dài và phức tạp.  
\end{itemize}

\textbf{Gợi ý hình minh hoạ:} Infographic “Hiện tại vs Tiềm năng”: dữ liệu hiện dùng (bán hàng, thời tiết) và dữ liệu tiềm năng (kinh tế, đối thủ, social media).

\subsection{Bài học Rút ra}

\begin{itemize}
  \item \textbf{Chất lượng dữ liệu là nền tảng}: không có dữ liệu sạch và giàu đặc trưng, mô hình sẽ không phát huy được sức mạnh.  
  \item \textbf{Khả năng giải thích quan trọng không kém độ chính xác}: chính XAI giúp nhà quản lý tin tưởng và sẵn sàng hành động dựa trên AI.  
  \item \textbf{Ứng dụng web thân thiện quyết định mức độ chấp nhận}: một giao diện dễ dùng như Streamlit đã biến mô hình từ “công cụ phòng lab” thành “người bạn đồng hành” của đội kinh doanh.  
\end{itemize}

\section{Các Hướng Phát triển Tương lai}

Một PoC thành công chưa đủ để trở thành một sản phẩm doanh nghiệp thực thụ.  
Điều quan trọng là phải có một \textbf{lộ trình phát triển rõ ràng}, 
để dự án không chỉ dừng lại ở mức thử nghiệm mà còn tạo ra tác động lâu dài.  

Dưới đây là bốn hướng phát triển được nhóm đề xuất:  

\subsection{Cải thiện Mô hình}

Mô hình LightGBM đã chứng minh hiệu quả, 
nhưng vẫn còn nhiều tiềm năng để khai thác thêm:

\begin{itemize}
  \item \textbf{Ensemble methods}: kết hợp nhiều mô hình (stacking, blending) để tăng độ chính xác và ổn định.  
  \item \textbf{Deep learning}: tích hợp các kiến trúc như LSTM hoặc Transformer – vốn rất mạnh với chuỗi thời gian dài và phức tạp.  
  \item \textbf{Online learning}: để mô hình có thể cập nhật liên tục từ dữ liệu mới, duy trì độ chính xác theo thời gian.  
\end{itemize}

\textbf{Gợi ý hình minh hoạ:} Biểu đồ so sánh “single model” vs “ensemble” vs “deep learning”.

\subsection{Mở rộng Kỹ thuật Tạo Đặc trưng}

Đặc trưng chính là “ngôn ngữ” mà mô hình sử dụng để hiểu dữ liệu.  
Do đó, việc mở rộng kỹ thuật tạo đặc trưng có thể tăng đáng kể sức mạnh dự báo:

\begin{itemize}
  \item \textbf{Tích hợp dữ liệu ngoài}: chỉ số kinh tế vĩ mô, dữ liệu mạng xã hội (social media sentiment), sự kiện đặc biệt (chiến dịch marketing lớn).  
  \item \textbf{Đặc trưng nâng cao}: áp dụng các kỹ thuật như Wavelets, Fourier Transform để phân tích chu kỳ ẩn trong dữ liệu.  
  \item \textbf{Đặc trưng liên cửa hàng (cross-store)}: mô hình hóa mối quan hệ giữa các cửa hàng để học tác động chéo.  
\end{itemize}

\textbf{Gợi ý hình minh hoạ:} Infographic các loại dữ liệu mở rộng: kinh tế, social media, sự kiện đặc biệt.

\subsection{Nâng cao Khả năng Giải thích}

SHAP là một bước tiến lớn, nhưng vẫn có thể đi xa hơn:

\begin{itemize}
  \item \textbf{Counterfactual explanations}: trả lời câu hỏi “Điều gì sẽ xảy ra nếu...?”, 
  ví dụ “Nếu tăng ngân sách marketing 10\%, doanh số có thể thay đổi thế nào?”.  
  \item \textbf{Causal inference}: phân tích quan hệ nhân quả, không chỉ dừng ở mức tương quan, 
  để xác định yếu tố nào thực sự gây ra thay đổi doanh số.  
\end{itemize}

Điều này biến mô hình dự báo thành một công cụ không chỉ giải thích quá khứ, 
mà còn gợi ý hướng đi cho tương lai.  


\subsection{Mở rộng Kinh doanh}

Để thực sự đi vào hoạt động, hệ thống cần tích hợp sâu hơn với quy trình doanh nghiệp:  

\begin{itemize}
  \item \textbf{API development}: xây dựng kiến trúc hướng dịch vụ (SOA), để các hệ thống khác dễ dàng gọi dự báo.  
  \item \textbf{Real-time predictions}: tích hợp dữ liệu streaming để dự báo tức thời, phục vụ quyết định nhanh.  
  \item \textbf{Advanced analytics}: từ dự báo chuyển thành gợi ý hành động, ví dụ đề xuất chiến lược tồn kho tối ưu.  
\end{itemize}

\textbf{Gợi ý hình minh hoạ:} Roadmap phát triển 4 hướng: Model – Feature – XAI – Business.

\subsection{Tầm nhìn dài hạn}

Nếu các bước trên được thực hiện, dự án sẽ chuyển đổi từ một \textbf{PoC kỹ thuật} 
thành một \textbf{nền tảng chiến lược} cho doanh nghiệp.  
Nó không chỉ trả lời câu hỏi \emph{“Doanh số ngày mai sẽ ra sao?”}, 
mà còn giúp doanh nghiệp hoạch định chiến lược dài hạn, 
ứng phó kịp thời với biến động thị trường, 
và tận dụng tối đa sức mạnh của dữ liệu.




\section{Kết luận}
Dự án chứng minh rằng AI có thể mang lại dự báo doanh số chính xác và minh bạch.  
Explainable AI (XAI) giúp xây dựng niềm tin, biến kết quả AI thành công cụ hỗ trợ ra quyết định thực sự trong kinh doanh.  

\vspace{0.5cm}
\noindent\textbf{Mã nguồn dự án:} \url{https://github.com/nguyenhads/sales_forecasting_xai}

\end{document}
