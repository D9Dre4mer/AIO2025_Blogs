\documentclass[12pt,a4paper]{article}
\usepackage[utf8]{inputenc}
\usepackage{hyperref}
\usepackage{minted}
\usepackage{enumitem}
\usepackage{geometry}
\geometry{margin=2.5cm}
\usepackage{pifont}

\ding{51} 
\ding{55}

% Encoding và font cho tiếng Việt
\usepackage{polyglossia}
\setmainlanguage{vietnamese}
\setotherlanguage{english}

\usepackage{fontspec}
\setmainfont{Times New Roman}
\setsansfont{Times New Roman}
\setmonofont{Times New Roman}[Scale=MatchLowercase]

% Toán học và ký hiệu
\usepackage{amsmath, amssymb, amsfonts}
\usepackage{pifont} % For \ding commands
\usepackage{bbding} % Alternative dingbat symbols

% Cấu hình trang
\usepackage{geometry}
\geometry{left=2.5cm, right=2.5cm, top=2.5cm, bottom=2.5cm}

% Header/Footer
\usepackage{fancyhdr}
\setlength{\headheight}{40 pt}
\pagestyle{fancy}
\fancyhead[LO,L]{GRID034 (AIO)}
\fancyhead[CO,C]{}
\fancyhead[RO,R]{\today}
\fancyfoot[CO,C]{\thepage}

% Hình ảnh
\usepackage{graphicx}
\usepackage{subcaption}

% Minted - mã lệnh có hỗ trợ tiếng Việt tốt hơn
\usepackage{minted}
\usepackage{float}

% Màu sắc cho minted
\usepackage{xcolor}
\definecolor{codebg}{rgb}{0.97,0.97,0.97}

% Cấu hình minted
\setminted{
    fontsize=\footnotesize,
    breaklines=true,
    breakanywhere=true,
    linenos=true,
    frame=lines,
    framesep=2mm,
    bgcolor=codebg
}

% Hyperref setup
\usepackage[hidelinks]{hyperref}
\hypersetup{
    colorlinks=true,
    linkcolor=blue,
    filecolor=magenta,
    urlcolor=cyan
}

% Các package bổ sung
\usepackage{parskip}
\usepackage{enumitem}
\usepackage{longtable}
\usepackage{booktabs}
\usepackage{makecell}
\usepackage{algorithm}
\usepackage{algpseudocode}
\usepackage{cancel}
\usepackage{titlesec}
\usepackage{blindtext}
\usepackage{datetime}
\usepackage{placeins}
\usepackage{array}
\usepackage{tabularx}
\usepackage{multirow}
\usepackage[table]{xcolor}
\usepackage[bottom]{footmisc}
\usepackage{tcolorbox}  % For the tcolorbox environment
\usepackage{tikz}
\usetikzlibrary{positioning, arrows.meta, shapes.geometric}

% Unicode special characters
\usepackage{newunicodechar}

% Custom command
\newcommand{\code}[1]{\texttt{#1}}

% Date format
\newdateformat{mydate}{\THEDAY/\THEMONTH/\THEYEAR}

% Section formatting
\titleformat{\section}{\large\bfseries}{\thesection.}{0.5em}{}

% Unicode characters
\newunicodechar{✔}{\checkmark}
\newunicodechar{❌}{\texttimes} % approximate
\newunicodechar{⇒}{\ensuremath{\Rightarrow}}

% Thông tin tài liệu
\author{}
\date{\mydate\today} % Gọi preamble từ file riêng

\begin{document}

\title{Dựng môi trường phát triển bằng Docker trên máy cá nhân}
\author{Võ Hoàng}
\date{}

\maketitle

\section*{Giới thiệu}
Bài viết này là một tutorial thực chiến giúp bạn thiết kế \textbf{môi trường phát triển (dev env)} dùng Docker ngay trên máy cá nhân. 
Mục tiêu: \textbf{nhanh, tái lập, ít ``cồng kềnh'', dễ debug}, và tương thích nhóm.

\section{Đối tượng \& mục tiêu}
\begin{itemize}
  \item \textbf{Đối tượng}: Dev backend/frontend, data/ML, sinh viên, team nhỏ cần môi trường đồng nhất.
  \item \textbf{Mục tiêu}:
  \begin{itemize}
    \item Chạy dự án local bằng \verb|docker compose up| với hot-reload.
    \item Tách biệt \textbf{dev} và \textbf{prod} bằng profiles.
    \item Chuẩn hoá secrets, volumes, network.
    \item Chạy tốt trên Windows (WSL2), macOS, Linux.
  \end{itemize}
\end{itemize}

\section{Kiến trúc tổng quan (host ↔ Docker)}
\begin{minted}{text}
Host OS (Windows/macOS/Linux)
  └─ (Tuỳ chọn) WSL2 (Windows)
      └─ Docker Engine / Docker Desktop / Rancher Desktop
          └─ Containers (app, db, cache, etc.)
              └─ Volumes (bind/named) + Networks
\end{minted}

\textbf{Gợi ý:}
\begin{itemize}
  \item Windows: WSL2 + Docker Desktop (dễ nhất). Nếu tối giản: cài Docker Engine rootless trong WSL2.
  \item macOS: Docker Desktop hoặc Colima (nhẹ, nhanh, M-series friendly).
  \item Linux: Cài Docker Engine chính chủ.
\end{itemize}

\section{Chuẩn bị theo hệ điều hành}

\subsection{Windows 10/11 (WSL2 + Docker Desktop)}
\begin{minted}{powershell}
wsl --install -d Ubuntu
wsl --set-default-version 2
\end{minted}
Sau đó cài Docker Desktop, bật WSL 2 based engine, để project code trong WSL FS để tránh I/O chậm.
Di chuyển dữ liệu Docker sang ổ D bằng thiết lập trong Docker Desktop hoặc \verb|wsl --export/--import|.

\subsection{macOS (Intel/M-series)}
\begin{minted}{bash}
brew install colima docker
colima start --cpu 4 --memory 8 --disk 50
\end{minted}

\subsection{Linux (Ubuntu/Debian)}
\begin{minted}{bash}
sudo apt-get update
sudo apt-get install -y ca-certificates curl gnupg lsb-release
# Add Docker’s official GPG key & repo
sudo apt-get install -y docker-ce docker-ce-cli containerd.io \
    docker-buildx-plugin docker-compose-plugin
sudo usermod -aG docker $USER
newgrp docker
\end{minted}

\section{Tiêu chuẩn hoá bố cục dự án}
\begin{minted}{text}
myapp/
  backend/
    app/
      main.py
    pyproject.toml
    Dockerfile
    Dockerfile.dev
  db/
    init.sql
  .env
  docker-compose.yml
  Makefile
  .dockerignore
  .editorconfig
  .gitattributes
  .devcontainer/
    devcontainer.json
\end{minted}

\textbf{Quy ước:} Bind mount cho mã nguồn (dev), named volume cho dữ liệu (db, cache), secrets qua \verb|.env|, 
profiles để tách dev/prod.

\section{Ví dụ : FastAPI + PostgreSQL (hot-reload)}

\subsection{main.py}
\begin{minted}{python}
from fastapi import FastAPI
from os import getenv
import psycopg

app = FastAPI()

@app.get("/")
def read_root():
    return {"msg": "Hello Docker Dev!", "env": getenv("APP_ENV", "dev")}
\end{minted}

\subsection{pyproject.toml}
\begin{minted}{toml}
[project]
name = "myapp-backend"
version = "0.1.0"
dependencies = [
  "fastapi",
  "uvicorn[standard]",
  "psycopg[binary]",
]
\end{minted}

\subsection{Dockerfile.dev}
\begin{minted}{docker}
FROM python:3.12-slim
WORKDIR /app
COPY pyproject.toml /app/
RUN pip install --no-cache-dir uv pip-tools && \
    pip install --no-cache-dir fastapi uvicorn[standard] psycopg[binary]
CMD ["uvicorn", "app.main:app", "--host", "0.0.0.0", "--port", "8000", "--reload"]
\end{minted}

\subsection{docker-compose.yml}
\begin{minted}{yaml}
version: "3.9"
name: myapp

services:
  api:
    build:
      context: ./backend
      dockerfile: Dockerfile.dev
    ports:
      - "8000:8000"
    env_file: .env
    volumes:
      - ./backend:/app:rw
    depends_on:
      - db
    profiles: ["dev"]

  db:
    image: postgres:16
    environment:
      POSTGRES_USER: ${POSTGRES_USER}
      POSTGRES_PASSWORD: ${POSTGRES_PASSWORD}
      POSTGRES_DB: ${POSTGRES_DB}
    volumes:
      - pgdata:/var/lib/postgresql/data
      - ./db/init.sql:/docker-entrypoint-initdb.d/init.sql:ro
    ports:
      - "5432:5432"

volumes:
  pgdata:
\end{minted}

\vspace{1em}
Chạy dev:
\begin{verbatim}
docker compose --profile dev up --build
\end{verbatim}

Mở: \url{http://localhost:8000}.

\section{Ví dụ 2: Next.js + Redis (hot-reload)}

\subsection{frontend/src/index.js}
\begin{minted}{javascript}
export default function Home() {
  return <h1>Hello Docker Dev with Next.js!</h1>
}
\end{minted}

\subsection{frontend/package.json}
\begin{minted}{json}
{
  "name": "myapp-frontend",
  "version": "1.0.0",
  "scripts": {
    "dev": "next dev",
    "build": "next build",
    "start": "next start"
  },
  "dependencies": {
    "next": "14.0.0",
    "react": "18.2.0",
    "react-dom": "18.2.0"
  }
}
\end{minted}

\subsection{frontend/Dockerfile.dev}
\begin{minted}{docker}
FROM node:20-slim
WORKDIR /app
COPY package.json package-lock.json* ./
RUN npm install
CMD ["npm", "run", "dev"]
\end{minted}

\subsection{docker-compose.override.yml (mở rộng)}
\begin{minted}{yaml}
services:
  frontend:
    build:
      context: ./frontend
      dockerfile: Dockerfile.dev
    ports:
      - "3000:3000"
    volumes:
      - ./frontend:/app
    depends_on:
      - redis
    profiles: ["dev"]

  redis:
    image: redis:7
    ports:
      - "6379:6379"
    volumes:
      - redisdata:/data

volumes:
  redisdata:
\end{minted}

Mở: \url{http://localhost:3000}. Hot reload hoạt động tự động.

\section{CI/CD nâng cao (GitHub Actions + Docker Hub)}
\subsection{.github/workflows/deploy.yml}
\begin{minted}{yaml}
name: deploy

on:
  push:
    branches: [main]

jobs:
  build-and-push:
    runs-on: ubuntu-latest
    steps:
      - uses: actions/checkout@v4
      - name: Login Docker Hub
        uses: docker/login-action@v2
        with:
          username: ${{ secrets.DOCKERHUB_USERNAME }}
          password: ${{ secrets.DOCKERHUB_TOKEN }}
      - name: Build & Push
        run: |
          docker build -t myorg/myapp-backend:latest ./backend
          docker build -t myorg/myapp-frontend:latest ./frontend
          docker push myorg/myapp-backend:latest
          docker push myorg/myapp-frontend:latest
\end{minted}

\section{Triển khai thử nghiệm (Prod mini)}

Chạy trên một server (VD: VPS):

\begin{verbatim}
git pull origin main
docker compose --profile prod up -d --build
\end{verbatim}

Muốn expose với domain + HTTPS:
\begin{itemize}
  \item Dùng Traefik hoặc NGINX reverse proxy.
  \item Dùng Caddy (tự động cấp TLS).
\end{itemize}

\subsection{Compose với Caddy}
\begin{minted}{yaml}
services:
  caddy:
    image: caddy:2
    ports:
      - "80:80"
      - "443:443"
    volumes:
      - ./Caddyfile:/etc/caddy/Caddyfile:ro
      - caddy_data:/data
      - caddy_config:/config

volumes:
  caddy_data:
  caddy_config:
\end{minted}

\subsection{Caddyfile mẫu}
\begin{minted}{text}
myapp.localhost {
    reverse_proxy frontend:3000
}
\end{minted}

\section{So sánh Docker Desktop vs thay thế}
\begin{center}
\begin{tabular}{|l|c|c|c|}
\hline
Tính năng & Docker Desktop & Colima (macOS) & Rancher Desktop (Win/mac) \\
\hline
GUI quản lý container & \ding{51} & \ding{55} & \ding{51} \\
Dễ cài trên Win/mac   & \ding{51} & \ding{55} & \ding{51} \\
Hỗ trợ Kubernetes     & \ding{51} & \ding{55} & \ding{51} \\
Dùng ít RAM hơn       & \ding{55} & \ding{51} & \ding{51} \\
\hline
\end{tabular}
\end{center}

\section{Best practices nâng cao}
\begin{itemize}
  \item Luôn viết multi-stage build để image nhỏ gọn.
  \item Tách dev/prod config.
  \item Dùng linter + CI để check Dockerfile.
  \item Dùng healthcheck để giám sát service.
  \item Cache dependencies hợp lý (COPY requirements.txt trước).
\end{itemize}


\section{Inclusion}

Qua toàn bộ hành trình trên, bạn đã thấy rằng Docker không chỉ là công cụ chạy container đơn lẻ 
mà còn là nền tảng để chuẩn hoá môi trường phát triển. Khi biết cách tổ chức dự án, 
tách cấu hình dev/prod, quản lý volumes, secrets, và sử dụng thêm các công cụ như 
Makefile, Dev Containers, CI/CD pipeline, bạn sẽ có:

\begin{itemize}
  \item Một môi trường dev nhất quán, không còn cảnh ``works on my machine''.
  \item Quy trình on-boarding nhanh chóng: chỉ cần \verb|git clone| và \verb|docker compose up|.
  \item Khả năng mở rộng sang full-stack (backend + frontend + database + cache) 
        chỉ với vài service trong Compose.
  \item Dễ dàng nâng cấp: từ local $\to$ staging/prod $\to$ Kubernetes mà không phải viết lại quá nhiều.
  \item Tích hợp liền mạch với các công cụ hiện đại: GitHub Actions, secret managers, GPU runtime cho AI/ML.
\end{itemize}

Điều quan trọng hơn cả: Docker giúp bạn tư duy theo hướng 
\textit{infrastructure as code} – coi môi trường dev cũng là một phần của dự án, 
được quản lý version control, test, và tối ưu như chính mã nguồn. Đây chính là nền tảng 
để tiến tới CI/CD chuyên nghiệp và triển khai microservices ở quy mô lớn.

Hãy thử áp dụng các mẫu cấu trúc trong bài viết cho dự án của bạn, tuỳ biến cho phù hợp. 
Khi đã quen, bạn sẽ thấy việc tạo một môi trường dev hoàn chỉnh chỉ mất vài phút – 
và lợi ích mang lại thì vô cùng lớn.

\end{document}
