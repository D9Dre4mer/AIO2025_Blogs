\section{Trực giác LIME}

\textbf{LIME }(\textit{Local Interpretable Model-agnostic Explanations}) cung cấp một \emph{mô hình thay thế cục bộ} $g$ để mô tả hành vi của mô hình gốc $f$ quanh một điểm quan tâm $x$ \cite{ribeiro2016lime}. Thay vì cố hiểu toàn bộ $f$, ta ``phóng to'' vào vùng lân cận của $x$ bằng một thước đo gần--xa, sinh các điểm nhiễu có trọng số theo độ gần, rồi khớp một mô hình đơn giản (thường là tuyến tính thưa) để suy luận đóng góp của đặc trưng.

\subsection{Định nghĩa và hàm mục tiêu}

Gọi $\pi_x(z)$ là trọng số lân cận (kernel) đo mức ``gần'' giữa $z$ và $x$, $L$ là mất mát đo chênh lệch dự đoán giữa $f$ và mô hình thay thế $g$, và $\Omega(g)$ là phạt độ phức tạp (để khuyến khích tính gọn). LIME tối ưu:
\[
g^\star \;=\; \arg\min_{g \in \mathcal{G}} \; L\big(f,g,\pi_x\big) \;+\; \Omega(g).
\]
Với phân loại nhị phân, $L$ thường là mất mát logistic hoặc bình phương có trọng số; với hồi quy, thường là MSE có trọng số:
\[
L\big(f,g,\pi_x\big) \;=\; \mathbb{E}_{z \sim \pi_x}\!\Big[w(z)\,\ell\!\big(f(z),g(z)\big)\Big],
\quad w(z)=\pi_x(z).
\]
Kernel phổ biến: $\pi_x(z)=\exp\!\big(-D(x,z)^2/\sigma^2\big)$, trong đó $D$ là khoảng cách được chọn theo miền (chuẩn hoá Euclid cho tabular, cosine cho text, v.v.). Tham số $\sigma$ điều khiển ``độ zoom'' của vùng cục bộ.

\subsection{Biểu diễn diễn giải \texorpdfstring{$x'$}{x'} và ánh xạ}

LIME làm việc trên một \emph{không gian diễn giải} $x'$ sao cho việc ``bật/tắt'' đặc trưng có ý nghĩa đối với con người:
\begin{itemize}
  \item \textbf{Bảng (tabular)}: $x'$ có thể là các đặc trưng đã rời rạc hoá/nhị phân hoá; một bit bằng 0 nghĩa là ``giữ cố định ở giá trị nền''.
  \item \textbf{Văn bản} (LIME-Text): $x'$ là sự hiện diện của token/bi-gram; tắt một token tương đương loại bỏ từ đó.
  \item \textbf{Ảnh} (LIME-Image): $x'$ là vector bật/tắt các \emph{superpixel}; tắt một superpixel tương đương đè nền xám/làm mờ vùng đó.
\end{itemize}
Ánh xạ $\phi:x' \mapsto x$ dựng lại một thể hiện trong không gian gốc để truy vấn $f$.

\subsection{Quy trình LIME theo từng bước}

\begin{enumerate}
  \item \textbf{Chọn điểm cần giải thích} $x$ và lớp đích (nếu phân loại đa lớp).
  \item \textbf{Sinh mẫu lân cận}: tạo $N$ biến thể $z_i' \sim q(\cdot\,|\,x')$ bằng cách bật/tắt các thành phần của $x'$ (với xác suất được hiệu chỉnh theo miền) rồi ánh xạ $z_i=\phi(z_i')$.
  \item \textbf{Gán trọng số lân cận}: $w_i=\pi_x(z_i)$ với khoảng cách $D$ phù hợp và kernel width $\sigma$.
  \item \textbf{Gọi mô hình gốc}: lấy $y_i=f(z_i)$ (xác suất/lô-git/giá trị dự đoán).
  \item \textbf{Khớp mô hình thay thế} $g$: thường dùng hồi quy tuyến tính thưa (\emph{K-LASSO}: chọn tối đa $K$ đặc trưng quan trọng rồi ước lượng lại bằng OLS) để tối thiểu $L\!+\!\Omega$.
  \item \textbf{Trình bày lời giải thích}: hệ số $\beta_j$ của $g$ (hoặc ảnh minh hoạ superpixel) cho biết mức đóng góp cục bộ của đặc trưng $j$ vào dự đoán tại $x$.
\end{enumerate}

\begin{takeaway}
LIME không cố ``đúng mọi nơi''; nó ưu tiên \emph{đúng quanh $x$}, với mức độ đúng được kiểm soát bởi kernel và số mẫu lân cận.
\end{takeaway}

\subsection{Lựa chọn tham số theo miền}

\paragraph{Kích thước vùng lân cận.}
Kernel width $\sigma$ quá nhỏ $\Rightarrow$ mẫu rất sát $x$, dễ quá khớp nhiễu; quá lớn $\Rightarrow$ $g$ nhìn thấy cả cấu trúc toàn cục, mất tính cục bộ. Thực hành: quét $\sigma$ trên lưới nhỏ (ví dụ $\{0.25,0.5,1.0\}$ cho dữ liệu đã chuẩn hoá) và chọn theo \emph{fidelity cục bộ} cao nhất kèm độ ổn định tốt.

\paragraph{Số mẫu $N$.}
Tăng $N$ giúp giảm phương sai ước lượng nhưng tốn chi phí gọi $f$. Với tabular thường đủ $N\in[1000,5000]$; text và image cần lớn hơn do chiều cao và phân phối rời rạc (có thể $N\ge 2000$).

\paragraph{Số đặc trưng hiển thị $K$.}
Chọn $K$ theo \emph{sparsity} mong muốn (thường 5--10) để đảm bảo người dùng đọc được. Có thể tự động quyết định bằng đường cong ``fidelity--sparsity''.

\paragraph{Chiến lược nhiễu.}
\begin{itemize}
  \item \textbf{Tabular}: nhiễu có điều kiện theo thống kê biên/đồng phương sai để tránh điểm \emph{off-manifold}. Tránh giả định độc lập hoàn toàn khi có tương quan mạnh.
  \item \textbf{Text}: loại token với xác suất theo tần suất hoặc theo phân phối có điều kiện (giữ ngữ cảnh cơ bản), tránh xoá quá nhiều làm mất nghĩa câu.
  \item \textbf{Ảnh}: dùng phân đoạn superpixel hợp lý (K=50--200). Granularity quá thô $\Rightarrow$ lời giải thích cục bộ không chính xác; quá mịn $\Rightarrow$ nhiễu thị giác và không ổn định.
\end{itemize}

\subsection{Đo lường fidelity và stability}

Fidelity cục bộ: Báo cáo $R^2$ có trọng số hoặc accuracy có trọng số của $g$ trên tập mẫu lân cận:
\[
R^2_w = 1 - \frac{\sum_i w_i\,(y_i - g(z_i))^2}{\sum_i w_i\,(y_i - \bar{y}_w)^2},
\quad \bar{y}_w=\frac{\sum_i w_i y_i}{\sum_i w_i}.
\]
Fidelity cao là điều kiện cần để tin cậy lời giải thích.

Stability: Lặp quá trình $T$ lần (khác seed, nhỏ nhiễu), đo Jaccard giữa tập top-$K$ đặc trưng: 
\[
\mathrm{Jac} = \frac{|S^{(a)}\cap S^{(b)}|}{|S^{(a)}\cup S^{(b)}|}.
\]
Có thể dùng Kendall $\tau$ cho thứ hạng hoặc độ lệch hệ số $\|\beta^{(a)}-\beta^{(b)}\|$.

Kiểm nghiệm ``faithfulness'': Tắt lần lượt các đặc trưng được giải thích là quan trọng và quan sát mức suy giảm dự đoán $f$ (trên $x$ hoặc lân cận của $x$). Suy giảm phù hợp thứ hạng. Lời giải thích có ý nghĩa nhân quả gần đúng ở mức cục bộ (không đồng nghĩa quan hệ nhân quả thực sự).

\subsection{Ví dụ nhỏ: LIME-Text}

Giả sử phân loại cảm xúc câu tiếng Việt. Với câu: ``\emph{phim khá chậm nhưng kết đẹp, diễn viên chính rất tự nhiên}'', LIME-Text có thể chọn top-$K$ từ quan trọng: \{\texttt{kết}, \texttt{đẹp}, \texttt{tự\_nhiên}\} với hệ số dương, và \{\texttt{chậm}\} với hệ số âm. Fidelity có trọng số $R^2_w \approx 0.86$ cho thấy mô hình tuyến tính thưa đã mô phỏng tốt quyết định của $f$ quanh câu này. Lặp 5 lần cho Jaccard trung bình $\ge 0.6$ là mức ổn định chấp nhận được để báo cáo.

\subsection{Bẫy thường gặp và cách khắc phục}

\begin{itemize}
  \item \textbf{Nhiễu ngoài đa tạp (off-manifold).} Điểm nhiễu không giống dữ liệu thật làm $f$ dự đoán thiếu tin cậy. \emph{Khắc phục:} nhiễu có điều kiện, rời rạc hoá hợp lý, dùng kỹ thuật sinh mẫu gần dữ liệu gốc.
  \item \textbf{Đặc trưng tương quan cao.} Hệ số tuyến tính dễ không ổn định. \emph{Khắc phục:} nhóm/ghép đặc trưng, giảm $K$, tăng $N$, dùng regularization mạnh hơn.
  \item \textbf{Kernel width không phù hợp.} Quá nhỏ $\Rightarrow$ quá khớp; quá lớn $\Rightarrow$ mất cục bộ. \emph{Khắc phục:} quét $\sigma$ và chọn theo đường cong fidelity--stability.
  \item \textbf{Ảnh: phân đoạn kém.} Superpixel quá thô hoặc quá mịn. \emph{Khắc phục:} chọn K vừa phải, thử nhiều seed phân đoạn, báo cáo độ ổn định.
  \item \textbf{Explanation hacking.} Tối ưu lời giải thích thay vì mô hình. \emph{Khắc phục:} khoá cấu hình trước, kiểm thử ngoài mẫu, và dùng tiêu chí độc lập (faithfulness).
\end{itemize}

\subsection{Checklist cấu hình LIME (thực chiến)}

\begin{itemize}
  \item Chọn miền và khoảng cách $D$ hợp lý; chuẩn hoá dữ liệu trước khi tính khoảng cách.
  \item Chọn $\sigma$ bằng quét lưới ngắn, tối ưu fidelity cục bộ kèm điều kiện stability tối thiểu.
  \item Đặt $N$ đủ lớn cho miền (tabular: $\ge 1000$; text/image: $\ge 2000$), theo ngân sách gọi $f$.
  \item Đặt $K$ để đảm bảo sparsity (5--10) và dễ đọc; xác nhận bằng người dùng mục tiêu.
  \item Báo cáo: $R^2_w$/accuracy có trọng số, Jaccard/Kendall cho stability, và kiểm nghiệm faithfulness bằng ablation.
\end{itemize}

\begin{takeaway}
LIME hiệu quả khi bạn cần \emph{dấu hiệu định lượng, gọn} về đặc trưng nào đang chi phối một dự đoán cụ thể. Hãy coi fidelity và stability là hai ``đèn báo'' bắt buộc trước khi tin dùng kết quả \cite{ribeiro2016lime,doshi2017rigorous}.
\end{takeaway}