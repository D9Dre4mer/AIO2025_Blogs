\begin{center}
    \Large\textbf{SQL in Data Analysis}
\end{center}

\begin{center}
    \Large\textit{Bùi Đức Xuân}
\end{center}

\section{CREATE DATABASE — Tạo Cơ Sở Dữ Liệu}

\begin{lstlisting}[language=SQL]
CREATE DATABASE LibraryDB;
\end{lstlisting}

\textbf{Tính năng:}
\begin{itemize}
  \item Tạo cơ sở dữ liệu mới tên là \texttt{LibraryDB}.
  \item Tuỳ chọn: \texttt{IF NOT EXISTS} để tránh lỗi nếu CSDL đã tồn tại.
\end{itemize}

\begin{lstlisting}[language=SQL]
CREATE DATABASE IF NOT EXISTS LibraryDB;
\end{lstlisting}

\section{CREATE TABLE — Tạo Bảng}

\begin{lstlisting}[language=SQL]
USE LibraryDB;

CREATE TABLE Books (
    BookID INT AUTO_INCREMENT PRIMARY KEY,
    Title VARCHAR(100),
    Author VARCHAR(100),
    Genre VARCHAR(50),
    PublishedYear INT,
    Available BOOLEAN
);
\end{lstlisting}

\textbf{Tính năng:}
\begin{itemize}
  \item \texttt{AUTO\_INCREMENT} — Tăng tự động giá trị \texttt{BookID}.
  \item \texttt{PRIMARY KEY} — Khoá chính, xác định duy nhất từng dòng.
  \item Kiểu dữ liệu gồm: \texttt{VARCHAR}, \texttt{INT}, \texttt{BOOLEAN}, v.v.
  \item Ràng buộc phổ biến: \texttt{NOT NULL}, \texttt{UNIQUE}, \texttt{DEFAULT}, \texttt{CHECK}.
\end{itemize}

\section{DROP DATABASE — Xoá Cơ Sở Dữ Liệu}

\begin{lstlisting}[language=SQL]
DROP DATABASE LibraryDB;
\end{lstlisting}

\textbf{Tính năng:}
\begin{itemize}
  \item Xoá hoàn toàn cơ sở dữ liệu và tất cả bảng bên trong.
  \item Tuỳ chọn: \texttt{IF EXISTS} để tránh lỗi nếu không tồn tại.
\end{itemize}

\section{INSERT DATA — Chèn Dữ Liệu}

\begin{lstlisting}[language=SQL]
INSERT INTO Books (Title, Author, Genre, PublishedYear, Available) VALUES
('1984', 'George Orwell', 'Dystopian', 1949, TRUE),
('To Kill a Mockingbird', 'Harper Lee', 'Fiction', 1960, TRUE),
('The Great Gatsby', 'F. Scott Fitzgerald', 'Classic', 1925, FALSE),
('Brave New World', 'Aldous Huxley', 'Dystopian', 1932, TRUE),
('The Catcher in the Rye', 'J.D. Salinger', 'Classic', 1951, FALSE);
\end{lstlisting}

\textbf{Tính năng:}
\begin{itemize}
  \item Chèn một hoặc nhiều dòng dữ liệu vào bảng.
  \item Cần khớp thứ tự cột hoặc ghi rõ tên cột.
  \item Có thể dùng giá trị mặc định hoặc truy vấn phụ.
\end{itemize}

\section{SELECT — Truy vấn Dữ Liệu}

\begin{lstlisting}[language=SQL]
SELECT * FROM Books;
\end{lstlisting}

\textbf{Tính năng:}
\begin{itemize}
  \item \texttt{*} chọn tất cả các cột.
  \item Có thể chọn cột cụ thể: \texttt{SELECT Title, Author FROM Books;}
  \item Có thể đặt bí danh: \texttt{SELECT Title AS TenSach;}
\end{itemize}

\section{WHERE — Điều Kiện Lọc}

\begin{lstlisting}[language=SQL]
SELECT * FROM Books
WHERE Genre = 'Dystopian';
\end{lstlisting}

\textbf{Tính năng:}
\begin{itemize}
  \item Lọc dòng theo điều kiện.
  \item Hỗ trợ các toán tử logic: \texttt{=, <, >, <=, >=, <>, AND, OR, NOT}
  \item Có thể sử dụng biểu thức: \texttt{WHERE PublishedYear + 10 < 2000}
\end{itemize}

\section{IN và BETWEEN}

\begin{lstlisting}[language=SQL]
-- IN
SELECT * FROM Books
WHERE Genre IN ('Classic', 'Fiction');

-- BETWEEN
SELECT * FROM Books
WHERE PublishedYear BETWEEN 1930 AND 1960;
\end{lstlisting}

\textbf{Tính năng:}
\begin{itemize}
  \item \texttt{IN} — Kiểm tra giá trị nằm trong danh sách.
  \item \texttt{BETWEEN} — Kiểm tra giá trị trong khoảng (bao gồm giới hạn).
  \item Có thể dùng \texttt{NOT IN}, \texttt{NOT BETWEEN}.
\end{itemize}

\section{IS NULL — Kiểm Tra Giá Trị Rỗng}

\begin{lstlisting}[language=SQL]
SELECT * FROM Books
WHERE Author IS NULL;
\end{lstlisting}

\textbf{Tính năng:}
\begin{itemize}
  \item Kiểm tra các giá trị \texttt{NULL} (không có dữ liệu).
  \item Dùng \texttt{IS NOT NULL} để tìm dữ liệu không rỗng.
\end{itemize}

\section{ORDER BY — Sắp Xếp}

\begin{lstlisting}[language=SQL]
SELECT * FROM Books
ORDER BY PublishedYear DESC;
\end{lstlisting}

\textbf{Tính năng:}
\begin{itemize}
  \item Sắp xếp kết quả theo một hoặc nhiều cột.
  \item ASC = tăng dần (mặc định), DESC = giảm dần.
  \item Có thể kết hợp nhiều tiêu chí sắp xếp.
\end{itemize}

\begin{lstlisting}[language=SQL]
SELECT * FROM Books
ORDER BY Genre ASC, PublishedYear DESC;
\end{lstlisting}

\section{LIMIT — Giới Hạn Số Dòng}

\begin{lstlisting}[language=SQL]
SELECT * FROM Books
ORDER BY PublishedYear DESC
LIMIT 3;
\end{lstlisting}

\textbf{Tính năng:}
\begin{itemize}
  \item Giới hạn số dòng trả về.
  \item Có thể dùng \texttt{LIMIT offset, count} để phân trang.
\end{itemize}

\begin{lstlisting}[language=SQL]
SELECT * FROM Books
LIMIT 5, 10;
\end{lstlisting}
5 là offset — bỏ qua 5 dòng đầu. 10 là số dòng sẽ lấy tiếp theo.

\section{LIKE và REGEXP — Tìm Kiếm Mẫu}

\begin{lstlisting}[language=SQL]
-- LIKE
SELECT * FROM Books
WHERE Title LIKE 'The%';

-- REGEXP
SELECT * FROM Books
WHERE Author REGEXP '^J';
\end{lstlisting}

\textbf{Tính năng:}
\begin{itemize}
  \item \texttt{LIKE} sử dụng ký tự đại diện:
    \begin{itemize}
      \item \texttt{\%} — khớp mọi chuỗi ký tự.
      \item \texttt{\_} — khớp một ký tự duy nhất.
    \end{itemize}
\end{itemize}
