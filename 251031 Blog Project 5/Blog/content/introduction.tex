\section*{Giới thiệu}
Blog này trình bày toàn bộ quá trình phát triển và mở rộng dự án \textbf{MLDockFlow}, xuất phát từ đề tài gốc \textit{Sales Prediction} trong Module 5 của chương trình AIO2025. Dự án ban đầu tập trung vào việc dự đoán giá nhà thông qua các mô hình hồi quy tuyến tính và phi tuyến, nhưng trong phiên bản mở rộng này, ta đã tích hợp các công cụ và kỹ thuật hiện đại để tiến tới một quy trình học máy hoàn chỉnh, tự động và có khả năng triển khai thực tế.

Nội dung blog được tổ chức thành các phần chính sau:

\begin{enumerate}
    \item \textbf{So sánh dự án gốc và phiên bản nâng cấp:} Phân tích chi tiết các nâng cấp từ dự án gốc (Project 5.1) lên MLDockFlow, bao gồm cấu trúc code, preprocessing, models, experiment tracking và deployment. Phần này giúp người đọc hiểu rõ sự tiến hóa của dự án.
    
    \item \textbf{Mục tiêu của dự án:} Giới thiệu mục đích của việc mở rộng project ban đầu thành hệ thống học máy có thể theo dõi, quản lý và triển khai tự động.
    
    \item \textbf{Tiến trình phát triển:} Tóm tắt quy trình ta đã thực hiện — từ giai đoạn nghiên cứu, xây dựng pipeline, thực nghiệm mô hình cho đến triển khai và kiểm thử.
    
    \item \textbf{Tính năng cải tiến và tối ưu hóa:} Trình bày các cải tiến kỹ thuật bao gồm stacking pipeline, tự động hóa huấn luyện và kiểm soát mô hình.
    
    \item \textbf{Phân tích kết quả và các lựa chọn:} So sánh hiệu năng giữa các mô hình, thảo luận các quyết định liên quan đến lựa chọn siêu tham số, độ chính xác và khả năng tổng quát hóa.
    
    \item \textbf{Hệ thống MLFLOW:} Mô tả cách ta sử dụng MLflow để theo dõi, ghi log, và quản lý các phiên bản thí nghiệm, giúp quy trình tái lập và minh bạch hơn.
    
    \item \textbf{Docker Integrations:} Giải thích cách đóng gói mô hình bằng Docker Compose, triển khai ứng dụng web và quản lý môi trường thực thi.
    
    \item \textbf{Future Works:} Đề xuất các hướng phát triển trong tương lai, bao gồm mở rộng pipeline cho các bài toán khác, cải thiện hiệu năng mô hình và tăng mức độ tự động hóa.
    
    \item \textbf{Tổng kết \& Bài học rút ra:} Nhìn lại toàn bộ dự án, nêu ra những bài học thực tế về teamwork, kỹ năng kỹ thuật và tư duy triển khai hệ thống học máy hoàn chỉnh.
\end{enumerate}

\vspace{1em}
\noindent
\textbf{Giá trị nhận được sau khi đọc Blog:}
\begin{itemize}
    \item Hiểu quy trình phát triển một dự án học máy từ giai đoạn nghiên cứu đến triển khai thực tế với MLOps.
    \item Nắm bắt cách tích hợp \textit{MLflow} và \textit{Docker} để hình thành một pipeline MLOps đơn giản mà hiệu quả.
    \item Học được cách thiết kế \textit{stacking pipeline} để kết hợp sức mạnh của nhiều mô hình.
    \item Biết cách ghi log có tổ chức và quản lý thí nghiệm với MLflow Tracking Server.
    \item Hiểu rõ cách containerize và triển khai mô hình học máy với Docker Compose.
    \item Phát triển tư duy hệ thống hóa trong các dự án AI hiện đại — nền tảng quan trọng cho mọi kỹ sư dữ liệu và học máy.
\end{itemize}
